%set the master document for easy compilation
%!TEX root = ../D3_5_3.tex

\paragraph{Component Requirements}

\begin{longtable}{p{.25\textwidth}p{.7\textwidth}}
\toprule
Component name			& CalcBrakingCurves\_Integration \\
\midrule
Link to SCADE model		& {\footnotesize \url{???}} \\
\midrule
SCADE designer			& Christian Stahl, TWT \\
\midrule
Description				& For each type of target a certain braking curve has to be calculated. This curve enables proactive monitoring of the train's speed. A reverse lookup on this braking curve indicates, where the train has to start braking given the current speed. The braking curve does not depend on the actual train status. As a consequence the braking curve stays constant over time. As a legitimate simplification the calculation of the braking curve is not extended after the estimated front end position of the train has been passed. \\
\midrule
Input documents	& 
Subset-026, Chapter 3.13.8.3: Emergency Brake Deceleration curves (EBD)\newline
Subset-026, Chapter 3.13.8.4: Service Brake Deceleration curves (SBD)\newline
Subset-026, Chapter 3.13.8.5: Guidance curves (GUI) \\
\midrule
Safety integrity level		& 4 \\
\midrule
Time constraints		& [If applicable description of time constraints, otherwise n/a] \\
\midrule
API requirements 		& [If applicable description of API requirements, otherwise n/a] \\
\bottomrule
\end{longtable}


\paragraph{Interface}

For an overview of the interface of this internal component we refer to the SCADE model (c.f.~link above) respectively the SCADE generated documentation.