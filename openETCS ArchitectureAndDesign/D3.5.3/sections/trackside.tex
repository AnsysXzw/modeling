%set the master document for easy compilation
%!TEX root = ../D3_5_3.tex

\section{ETCS Messaging: TrackMessages}

\subsection{Component Requirements}

\begin{longtable}{p{.25\textwidth}p{.7\textwidth}}
\toprule
Component name			& TrackMessages::Read\_P005 \\
\midrule
Link to SCADE model		& {\footnotesize \url{https://github.com/openETCS/modeling/tree/master/model/Scade/System/ObuFunctions/ETCS_Messaging/TrackMessages}} \\
\midrule
SCADE designer			& Jakob G\"artner, LEA Railergy \newline
Mairamou Haman Adji, LEA Railergy\\
\midrule
Description				& TrackMessages is a library containing functionality to:\newline
\begin{itemize}
\item Transport TrainToTrack and TrackToTrain messages and packets using a compressed format which is conceptually close to the ETCS language as defined in Subset-026
\item Compress trackside information and decompress it in the onboard unit, taking into account different baseline versions and providing transparent translation.
\item Compress trainside information and decompress it in the trackside simulation models, taking into account different baseline versions and providing transparent translation. \newline
\end{itemize}

As TrackMessages is a library with various components supporting all packets and messages defined in Subset-026, we have selected one exemplary function to document the concept. As only the packet/ message- related functionality is specific, this approach will allow a first understanding of the concept and the related interfaces. For a full discussion of the library, refer to the [specifc chapter? document?]\newline
The function Read\_P005 extracts a packet 5 (Gradient Profile) from the compressed packets data flow, if present. It translates the integer- coded compressed data with the help of the metadata in the header section of the CompressedPackets\_T formatted data flow. After performing variable-level translation and exception detection, a baseline-3 conformal packet 5 is available for use within the relevant OBU functions.
\\

\midrule
Input documents	& 
Subset-026, Chapter 6\newline
Subset-026, Chapter 7\newline
Subset-026, Chapter 8\newline
\newline
The objective of this component (the full TrackMessages library) is to provide a full formalisation of above chapters in Subset-026\\
\midrule
Safety integrity level		& 4 \\
\midrule
Time constraints		& n/a (for the provided example function) \\
\midrule
API requirements 		& In the demonstrator context, the API is fully defined on SCADE model level. For integration with external systems (BTM, Radio, Subset-076 or Subset-94), additional conversion to/ from bit-level representation will be required\\
\bottomrule
\end{longtable}


\subsection{Interface}

An overview of the interface of component [component name] is shown in Figure~\ref{f:trackside_interface}. The inputs and outputs are described in detail in Section~\ref{s:trackside_inputs} respectively \ref{s:trackside_outputs}.

\begin{figure}
\center
\missingfigure{[Put SysML diagram of component here]}
\caption{Component SysML diagram}\label{f:trackside_interface}
\end{figure}


\subsubsection{Inputs}\label{s:trackside_inputs}

\paragraph{Message\_In}

\begin{longtable}{p{.25\textwidth}p{.7\textwidth}}
\toprule
Input name				&Message\_In \\
\midrule
Description				& Message\_In takes the compressed track-to-train messages that have either been compressed by the trackside simulation components of the TrackMessages library, or have been filled by the API. All packets that are part of the same message are transmitted within one cycle of the model's execution. Message\_IN is taking the compressed packet information from the track to train dataflow. \newline
  \\
\midrule
Source					& Manage\_TrackSideInformation\_Integration \\ 
\midrule
Type					& Common\_Types\_Pkg::CompressedPackets\_T \\
\midrule
Valid range of values 	& n/a\newline
The consistency of the metadata is checked at the input side.\newline
The ranges of the transported variables are checked at the conversion step (from integer format to SRS- conformal format)
 \\
\midrule
Behaviour when value is at boundary	& n/a \\
\midrule
Behaviour for values out of valid range	& n/a \newline \newline
The content of this input is not checked, as any issues will be found at conversion level. If the metadata are not matching the search criteria the packet will be considered as non existent and will therefore be ignored. 
 \\
\bottomrule
\end{longtable}




\subsubsection{Outputs}\label{s:trackside_outputs}

\paragraph{received}

\begin{longtable}{p{.25\textwidth}p{.7\textwidth}}
\toprule
Output name				& received \\
\midrule
Description				& Flag to indicate reception of a packet 5 from trackside in the current cycle \\
\midrule
Destination				& Any calling component\\ 
\midrule
Type					& bool \\
\midrule
Valid range of values	&
[true | false]
 \\
\midrule
Behaviour when value is at boundary	& n/a \\
\midrule
Behaviour for values out of valid range	& n/a \\
\midrule
Behaviour when value is erroneous, absent or unwanted (i.e. spurious) & n/a\\
\bottomrule
\end{longtable}


\paragraph{P005\_OBU\_out}

\begin{longtable}{p{.25\textwidth}p{.7\textwidth}}
\toprule
Output name				& P005\_OBU\_out \\
\midrule
Description				& Gradient Profile (Packet 5) according to 7.4.2.2 \\
\midrule
Destination				& Any calling operator\\ 
\midrule
Type					& TM::P005\_OBU\_T\\
\midrule
Valid range of values	& TM::P005\_OBU\_T is a complex data type. Values are given for each element.\newline Format is: Type Name: range/ list of values
\begin{itemize}
\item bool valid: [true | false]
\item q\_dir Q\_DIR:\newline  [Q\_DIR\_Both\_directions |\newline Q\_DIR\_Nominal |\newline Q\_DIR\_Reverse]
\item l\_packet L\_PACKET: (0-8191)
\item q\_scale Q\_SCALE: \newline[ENUM\_Q\_SCALE\_10cm |\newline ENUM\_Q\_SCALE\_1m |\newline ENUM\_Q\_SCALE\_10m]
\item n\_iter N\_ITER: (0-33)\newline \emph{(Remark: start section from the original packet is integrated into the list of sections)}
\end{itemize}
The structured element sections is an array of type P005\_section\_enum\_T. For each element, the valid range of values is as follows:
\begin{itemize}
\item bool valid: [true | false]\newline \emph{(Remark: Check for consistency with the value of n\_iter)}
\item d\_link D\_LINK: (0-32767)
\item q\_newcountry Q\_NEWCOUNTRY:\newline [TM\_conversions::ENUM\_Q\_NEWCOUNTRY\_same |\newline TM\_conversions::ENUM\_Q\_NEWCOUNTRY\_not\_same]
\item nid\_c NID\_C: (0-1023)
\item nid\_bg NID\_BG: (0-16383)
\item q\_linkorientation Q\_LINKORIENTATION:\newline [TM\_conversions::ENUM\_Q\_LINKORIENTATION\_reverse|\newline TM\_conversions::ENUM\_Q\_LINKORIENTATION\_nominal]
\item q\_linkreaction Q\_LINKREACTION:\newline [TM\_conversions::ENUM\_Q\_LINKREACTION\_Train\_trip |\newline TM\_conversions::ENUM\_Q\_LINKREACTION\_Apply\_service\_brake\newline TM\_conversions::ENUM\_Q\_LINKREACTION\_No\_Reaction]
\item q\_locacc Q\_LOCACC: (0-63)
\end{itemize}

\emph{Only an output structure with the structured element "valid" set to "true" is to be considered as received. If this field is set to true, the Output 1 (received) must equally be set to "true".}

 

 \\
\midrule
Behaviour when value is at boundary	& n/a \\
\midrule
Behaviour for values out of valid range	& The component is prepared for the upcoming error/exception handling concept. An error flag is, at the moment, raised internally if any of the compressed input values is out of range. A hierarchical error processing is foreseen.\newline

\emph{The types that have been defined in the package S026\_7 do not provide any default/ invalid value. The following fields are therefore set to an arbitrary value upon reception of an out-of-range value from track side, and the internal error flag is raised:}
\begin{itemize}
\item q\_dir Q\_DIR:\newline  set to: Q\_DIR\_Both\_directions
\item q\_scale Q\_SCALE: \newline set to: ENUM\_Q\_SCALE\_10cm 
\item q\_newcountry Q\_NEWCOUNTRY:\newline set to:[TM\_conversions::ENUM\_Q\_NEWCOUNTRY\_same |\newline TM\_conversions::ENUM\_Q\_NEWCOUNTRY\_not\_same]
\item q\_newcountry Q\_NEWCOUNTRY:\newline set to: TM\_conversions::ENUM\_Q\_NEWCOUNTRY\_not\_same
\item q\_linkorientation Q\_LINKORIENTATION:\newline set to: TM\_conversions::ENUM\_Q\_LINKORIENTATION\_reverse
\item q\_linkreaction Q\_LINKREACTION:\newline set to: TM\_conversions::ENUM\_Q\_LINKREACTION\_Train\_trip
\end{itemize}

 \\
\midrule
Behaviour when value is erroneous, absent or unwanted (i.e. spurious) & n/a \\
\bottomrule
\end{longtable}


\subsection{Sub Components}

\subsubsection{Read\_Packets}
%set the master document for easy compilation
%!TEX root = ../D3_5_3.tex

%\section{ETCS Messaging: TrackMessages}

\subsection{Component Requirements}

\begin{longtable}{p{.25\textwidth}p{.7\textwidth}}
\toprule
Component name			& TM\_lib\_internal::RECV\_ReadPackets \\
\midrule
Link to SCADE model		& {\footnotesize \url{https://github.com/openETCS/modeling/tree/master/model/Scade/System/ObuFunctions/ETCS_Messaging/TrackMessages}} \\
\midrule
SCADE designer			& Jakob G\"artner, LEA Railergy\\
\midrule
Description				& RECV\_ReadPackets extracts packet data information and raw compressed packet data from the compressed packets data flow, using filter criteria provided through parameter inputs:\newline
\begin{itemize}
\item NID\_PACKET: search for a specific packet
\item Version Number: search for a specific version number
\item Q\_DIR: search for packets that are only valid for a specific direction
\item Serial number: search for a specific packet instance, if several instances of a given packet type exist
\item F\_Version: Flag to decide whether to evaluate or ignore packet version information.
\item F\_id: Flag whether to evaluate or ignore packet serial number information.\newline
\end{itemize}

The operator TM\_lib\_internal::RECV\_ReadPackets takes a set of parameter data to 

\begin{itemize}
\item 1. Search the metadata of the compressed packets data flow using the provided parameters to determine if a matching packet is contained in any given cycle
\item 2. Output the flag "received" exactly in any cycle a matching packet is found
\item 3. Output an array of compressed packet data that is filled with the data from the identified packet
\end{itemize}


\\

\midrule
Input documents	& 
Subset-026, Chapter 7\newline
\newline
This function is not directly traceable to Subset-026, but is built from derived requirements\\
\midrule
Safety integrity level		& 4 \\
\midrule
Time constraints		& n/a  \\
\midrule
API requirements 		& In the demonstrator context, the API is fully defined on SCADE model level. For integration with external systems (BTM, Radio, Subset-076 or Subset-94), additional conversion to/ from bit-level representation will be required\\
\bottomrule
\end{longtable}
%-------------------------------------------------------------------------------------------------------------------------
%-------------------------------------------------------------------------------------------------------------------------
%-------------------------------------------------------------------------------------------------------------------------


\subsection{Interface}

An overview of the interface of component [component name] is shown in Figure~\ref{f:trackside_interface}. The inputs and outputs are described in detail in Section~\ref{s:trackside_inputs} respectively \ref{s:trackside_outputs}.

\begin{figure}
\center
\missingfigure{[Put SysML diagram of component here]}
\caption{Component SysML diagram}\label{f:trackside_interface}
\end{figure}

%-------------------------------------------------------------------------------------------------------------------------
%-------------------------------------------------------------------------------------------------------------------------


\subsubsection{Inputs}\label{s:trackside_inputs}

%-------------------------------------------------------------------------------------------------------------------------


\paragraph{Message\_In}

\begin{longtable}{p{.25\textwidth}p{.7\textwidth}}
\toprule
Input name				&Message\_In \\
\midrule
Description				& Message\_In takes the compressed track-to-train messages that have either been compressed by the trackside simulation components of the TrackMessages library, or have been filled by the API. All packets that are part of the same message are transmitted within one cycle of the model's execution. Message\_IN is taking the compressed packet information from the track to train dataflow. \newline
  \\
\midrule
Source					& Manage\_TrackSideInformation\_Integration through any calling operator (TM::Read\_Pxxx) \\ 
\midrule
Type					& Common\_Types\_Pkg::CompressedPackets\_T \\
\midrule
Valid range of values 	& n/a\newline
The consistency of the metadata is checked at the input side.\newline
The ranges of the transported variables are checked at the conversion step (from integer format to SRS- conformal format)
 \\
\midrule
Behaviour when value is at boundary	& n/a \\
\midrule
Behaviour for values out of valid range	& n/a \newline \newline
The content of this input is not checked, as any issues will be found at conversion level. If the metadata are not matching the search criteria the packet will be considered as non existent and will therefore be ignored. 
 \\
\bottomrule
\end{longtable}

%-------------------------------------------------------------------------------------------------------------------------


\paragraph{PacketID}

\begin{longtable}{p{.25\textwidth}p{.7\textwidth}}
\toprule
Input name				&PacketID \\
\midrule
Description				& PacketID defines the criteria used to determine whether the compressed packets contain a packet of interest. \newline
The information is position coded into an integer:\newline
The format is PPPDVVSSS.
\begin{itemize}
\item PPP: NID\_PACKET according to 7.5.1.93. 3-digit format, leading zeros are omitted.
\item D: Q\_DIR according to 7.5.1.103. 1- digit integer representing the decimal conversion of the binary values defined in 7.5.1.103
\item VV: M\_VERSION according to 7.5.1.79. 2- digit integer representing the decimal conversion of the binary values defined in 7.5.1.79
\item SSS: Serial number: search for a specific packet instance, if several instances of a given packet type exist. For packets where this is relevant, the serial number is used to identify a packet instance according to SRS definition, for example for NID\_TSR.
\end{itemize}

  \\
\midrule
Source					& Calling function  (TM::Read\_Pxxx). This input is realised as a SCADE hidden input, allowing the calling function to provide the information as a parameter \\ 
\midrule
Type					& int \\
\midrule
Valid range of values 	& 
\begin{itemize}
\item PPP: NID\_PACKET 0-999. Values that are not contained in the list of packets 7.4.1.1 are ignored.
\item D: 0-3. Values that are not defined in 7.5.1.103 are ignored
\item VV: 0-99. Values that are not defined in 7.5.1.79 are ignored
\item SSS:0-999
\end{itemize}
.\newline

 \\
\midrule
Behaviour when value is at boundary	& n/a \\
\midrule
Behaviour for values out of valid range	& n/a \newline \newline
As the metadata are automatically encoded using the values for NID\_PACKET, Q\_DIR, M\_VERSION and the serial number when a packet is compressed at trackside, out-of-range values may lead to erroneous identification of packet data. \newline
A static validation of the parameters used by the calling function is required. 
 \\
\bottomrule
\end{longtable}

%-------------------------------------------------------------------------------------------------------------------------


\paragraph{F\_version}

\begin{longtable}{p{.25\textwidth}p{.7\textwidth}}
\toprule
Input name				&F\_version \\
\midrule
Description				& F\_version is a flag. If set to true, version information will be taken into account when looking for a packet.\newline

  \\
\midrule
Source					& Calling function  (TM::Read\_Pxxx). This input is realised as a SCADE hidden input, allowing the calling function to provide the information as a parameter \\ 
\midrule
Type					& bool \\
\midrule
Valid range of values 	& [true | false]
.\newline

 \\
\midrule
Behaviour when value is at boundary	& n/a \\
\midrule
Behaviour for values out of valid range	& n/a  \newline
 \\
\bottomrule
\end{longtable}

%-------------------------------------------------------------------------------------------------------------------------

\paragraph{F\_id}

\begin{longtable}{p{.25\textwidth}p{.7\textwidth}}
\toprule
Input name				&F\_id \\
\midrule
Description				& F\_id is a flag. If set to true, serial number information will be taken into account when looking for a packet.\newline

  \\
\midrule
Source					& Calling function  (TM::Read\_Pxxx). This input is realised as a SCADE hidden input, allowing the calling function to provide the information as a parameter \\ 
\midrule
Type					& bool \\
\midrule
Valid range of values 	& [true | false]
.\newline

 \\
\midrule
Behaviour when value is at boundary	& n/a \\
\midrule
Behaviour for values out of valid range	& n/a  \newline

 \\
\bottomrule
\end{longtable}

%-------------------------------------------------------------------------------------------------------------------------
%-------------------------------------------------------------------------------------------------------------------------

\subsubsection{Outputs}\label{s:trackside_outputs}


\paragraph{Data}

\begin{longtable}{p{.25\textwidth}p{.7\textwidth}}
\toprule
Output name				& Data \\
\midrule
Description				& Contents of the packet, in integer array format. The first n elements of this array contain the data, trailing 0 that are beyond the length of the packet described in the metadata can be ignored. \\
\midrule
Destination				& Any calling operator\\ 
\midrule
Type					& Common\_Types\_Pkg::CompressedPacketData\_T\\
\midrule
Valid range of values	& n/a 

 \\
\midrule
Behaviour when value is at boundary	& n/a \\
\midrule
Behaviour for values out of valid range	& n/a\newline


 \\
\midrule
Behaviour when value is erroneous, absent or unwanted (i.e. spurious) & n/a \\
\bottomrule
\end{longtable}

%-------------------------------------------------------------------------------------------------------------------------


\paragraph{Metadata}

\begin{longtable}{p{.25\textwidth}p{.7\textwidth}}
\toprule
Output name				& Metadata \\
\midrule
Description				& Raw Metadata for the found packet \\
\midrule
Destination				& Any calling operator\\ 
\midrule
Type					& Common\_Types\_Pkg::MetadataElement\_T\\
\midrule
Valid range of values	& n/a 

 \\
\midrule
Behaviour when value is at boundary	& n/a \\
\midrule
Behaviour for values out of valid range	& n/a\newline


 \\
\midrule
Behaviour when value is erroneous, absent or unwanted (i.e. spurious) & n/a \\
\bottomrule
\end{longtable}

%-------------------------------------------------------------------------------------------------------------------------

\paragraph{received}

\begin{longtable}{p{.25\textwidth}p{.7\textwidth}}
\toprule
Output name				& received \\
\midrule
Description				& Flag to indicate reception of a packet 5 from trackside in the current cycle \\
\midrule
Destination				& Any calling component\\ 
\midrule
Type					& bool \\
\midrule
Valid range of values	&
[true | false]
 \\
\midrule
Behaviour when value is at boundary	& n/a \\
\midrule
Behaviour for values out of valid range	& n/a \\
\midrule
Behaviour when value is erroneous, absent or unwanted (i.e. spurious) & n/a\\
\bottomrule
\end{longtable}



%\subsection{Sub Components}

%\subsubsection{Name\_of\_Subcomponent}
%\input{sections/Name\_of\_Subcomponent.tex}

\subsubsection{Extract Packet 5}
%set the master document for easy compilation
%!TEX root = ../D3_5_3.tex

%\section{ETCS Messaging: TrackMessages}

\subsubsection{Component Requirements}

\begin{longtable}{p{.25\textwidth}p{.7\textwidth}}
\toprule
Component name			& TM\_conversions::trackside.C\_P005\_compr\_onboard \\
\midrule
Link to SCADE model		& {\footnotesize \url{https://github.com/openETCS/modeling/tree/master/model/Scade/System/ObuFunctions/ETCS_Messaging/TrackMessages}} \\
\midrule
SCADE designer			& Jakob G\"artner, LEA Railergy\\
\midrule
Description				& If a matching packet 5 has been received, TM\_conversions::trackside.C\_P005\_compr\_onboard: takes the compressed packet data and converts them to an SRS conformal onboard packet format. Trailing 0 beyond the valid length of the packet are ignored. \newline

\\

\midrule
Input documents	& 
Subset-026, Chapter 7\newline
\\
\midrule
Safety integrity level		& 4 \\
\midrule
Time constraints		& n/a  \\
\midrule
API requirements 		& n/a\\
\bottomrule
\end{longtable}
%-------------------------------------------------------------------------------------------------------------------------
%-------------------------------------------------------------------------------------------------------------------------
%-------------------------------------------------------------------------------------------------------------------------




%\subsubsection{Name\_of\_Subcomponent}
%\input{sections/Name\_of\_Subcomponent.tex}