\documentclass{template/openetcs_article}
\usepackage[utf8x]{inputenc}
\usepackage{color}
\usepackage{lipsum,url}
\graphicspath{{./template/}{.}{./images/}}
\begin{document}
\frontmatter
\project{openETCS}

%Please do not change anything above this line
%============================
% The document metadata is defined below

%assign a report number here
\reportnum{OETCS/WP3/DOW}

%define your workpackage here
\wp{Work-Package 3: ``Modelling''}

%set a title here
\title{openETCS Modelling WP (WP3) Description of Work}

%set a subtitle here
%\subtitle{Revision}

%set the date of the report here
\date{October 2013}
\date{\today}

%define a list of authors and their affiliation here

\author{Stanislas Pinte}

\affiliation{T3.5 \& T3.6 Task Leader}

\author{WP3 participants}

\affiliation{OpenETCS}
    
% define the coverart
\coverart[width=350pt]{chart}

%define the type of report
\reporttype{Description of work}

\begin{abstract}
%define an abstract here
This document contains the description of work planned for WP3.  This revision is necessary, taking into account all the work that has been done until now in the project.  

The revised DoW consists of four chapters for the four Tasks of WP3.

\end{abstract}

%=============================
%Do not change the next three lines
\maketitle
\tableofcontents
%\listoffiguresandtables
\newpage
%=============================

% The actual document starts below this line
%=============================


%Start here


% Makes Marginpars easier to read
\setlength{\marginparwidth}{1in}
\let\oldmarginpar\marginpar
\renewcommand\marginpar[1]{\-\oldmarginpar[\raggedleft\scriptsize #1]%
{\raggedright\scriptsize #1}}

\newcommand{\oldtext}[1]{{Old: \scriptsize #1}}

\newenvironment{inoutput}
{\vspace{2mm}
\noindent
\begin{tabular}{|r|p{.68\linewidth}|l|}
\hline}
{
\hline
\end{tabular}}

\section*{Introduction}

TODO

%=======================================================================
\section{Participants and Planning}
%=======================================================================

%----- Here we include screenshot of up-to-date xls spreadsheet 
%----- Once confirmed after Braunschweig meeting

TODO: to be completed after Braunschweig meeting

%=======================================================================
\section{T3.5 SSRS and System Specifications Model(s)}
%=======================================================================

%----- NOTE: this section should be revised by the SSRS task leader
%----- NOTE: to clarify: should the SSRS task leader become the T3.5 task leader?
%----- NOTE: maybe the SSRS task belongs to another WP (WP2?) and task 3.5 only includes SysML system models

The SSRS will be composed of two parts: 1/ the functional decomposition and the API (architecture) and 2/ the requirements. 

The functional decomposition \& API (architecture) is a functional breakdown of the subsystem. It makes explicit the boundaries of the onboard subsystem itself, and also provides the internal functional allocations (architecture) of this subsystem. This internal decomposition \& API (architecture) is composed of functions and flows of data between these functions. From the API we will derive the limits of this System. 

All the objects described will be unambiguously named (in particular I\/O) in a data dictionnary. 

This functional decomposition will be described using a semi-formal language.

This allocations (architecture) is useful for the following reasons:
•	it makes the system easier to maintain;
•	it provides the boundaries of the system;
•	it eases the safety analyses and the V\&V (with the internal cut out, and the unambiguous flows of data);
•	it also helps  with modeling by providing some structure.

The second part of the SSRS is the requirements list. The requirements from the SRS are allocated toward the functions of the SSRS (the architecture), possibly split and rewritten in order to restrict their scope to these functions (of course, traceability is mandatory). They are also rewritten in order to match the objects named in the architecture (in particular internal and external I\/O). The requirements are provided in natural language (even if the objects are unambiguously named). The formalization layer is coming below the SSRS, with the model.
 
The architecture objects (functions, streams...) and the requirements are tagged Vital\/Non Vital.

TODO: to be completed by SSRS task leader

%=======================================================================
\section{T3.6 Functional Model(s)}
%=======================================================================

In this task, the WP3 participants shall model the functions of the Subset-026 and SSRS/System Specifications models using the various toolchains 
identified in the document D7.1.

%--- TODO: insert here Figure 5 from D7.1

WP3 participants shall group themselves together according to the toolchain used for modelling:

\begin{enumerate}
	\item SCADE toolchain
	\item ERTMSFormalSpecs toolchain
	\item B toolchain
\end{enumerate}

Each group shall deliver the following artifacts:

\begin{enumerate}
	\item A toolchain documentation
	\item A document describing the modelling process followed
	\item The model
	\item The model tests
	\item If available under OSS license, the toolchain used to model and test the model
	\item A traceability report, tracing the model back to Subset-026 and SSRS/System Specifications models
	\item A report of coverage of the specifications by the model	
	\item A report of coverage of specifications by tests	
	\item A report of the execution of tests against the model
	\item A report listing all modelling issues encountered during modelling and/or testing
\end{enumerate}

At regular intervals, each modelling group shall meet in order to:

\begin{itemize}
	\item Exchange and discuss modelling issues
	\item Work on strategies to connect the model for integration and cross-testing
	\item Synchronize with WP4, in order to ensure the models meet necessary requirements from WP4 to be verified and validated
\end{itemize}

%---- TODO: according to Uwe Steinke, different teams should model the same function, and synchronize their results.

%=======================================================================
\section{T3.7 System Architecture Model(s)}
%=======================================================================

%--- The architecture model(s) of ETCS system.
%--- Specification of the API of the platform on which OpenETCS code runs.

TODO: to be completed by T3.7 leader or WP3 leader

%=======================================================================
\section{T3.8 Platform independent executable model (PIEM)}
%=======================================================================

%--- Source code generation.

TODO: to be completed by T3.8 leader or WP3 leader

\end{document}
Local Variables:
ispell-local-dictionary: "english"
End:
