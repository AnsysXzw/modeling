%set the master document for easy compilation
%!TEX root = ../D3_5_2.tex

\chapter{Introduction}

A primary goal of the openETCS ITEA2 project is to provide a formal specification and a non-vital reference implementation of an ETCS onboard unit (OBU) according to the specification given in the so called Subset-026 \cite{subset-026} defined the European Railway Agency (ERA). 

This deliverable, i.e.~D3.5.x, describes the architecture and design specification of the openETCS onboard (OBU) model. As the development of the OBU model is done iteratively according to a SCRUM process, the last digit of the deliverable identifier, i.e.~x, denotes the current iteration of the model. It should be considered as a complement to the following project outcomes respectively deliverables:
\begin{itemize}
\item the corresponding SysML and SCADE models, available at \url{https://github.com/openETCS/modeling/tree/master/model/Scade/System},
\item the corresponding functional design description, i.e.~D3.6.x, and
\item the documentation of the generic openETCS Application Programming Interface (API), available at \url{https://github.com/openETCS/modeling/blob/master/API/description/api-description.pdf}.
\end{itemize}


\section{Input Documents}

The following documents have been the basis for the analysis, functional decomposition, and design of the openETCS OBU functions:
\begin{itemize}
\item ERA Subset-026 \cite{subset-026}
\item ERA TSI CCS Documents
\item openETCS API documentation, available at \url{https://github.com/openETCS/modeling/blob/master/API/description/api-description.pdf}
\item openETCS requirements, i.e.~D2.1-9, available at \url{https://github.com/openETCS/requirements/tree/master/Reference}
\todo[fancyline]{list has to be completed}
\end{itemize}