\documentclass{template/openetcs_report}
% Use the option "nocc" if the document is not licensed under Creative Commons
%\documentclass[nocc]{template/openetcs_article}
\usepackage{lipsum,url}
\usepackage{supertabular}
\usepackage{multirow}
\usepackage{color, colortbl}
\usepackage{hyperref}
\usepackage{listings}
\usepackage{makeidx}
%\usepackage{amsthm}
\newtheorem{remark}{Remark}
\definecolor{gray}{rgb}{0.8,0.8,0.8}
\usepackage[modulo]{lineno}
\graphicspath{{./template/}{.}{./images/}}
\begin{document}
\frontmatter
\project{openETCS}

\newcommand{\define}[1]{\index{#1}\emph{#1}}

%Please do not change anything above this line
%============================

%user specified macros
%\newenvironment{activity}[2][planned]
	{\begin{tabular}{p{0.25\textwidth}@{\hspace{0.05\textwidth}}p{0.7\textwidth}}
			\multicolumn{2}{p{\textwidth}}{\colorbox{black}{\begin{minipage}{1.1cm}\begin{center}\textsc{\footnotesize \textcolor{white}{#1}}\end{center}\end{minipage}}~~\textbf{#2}}\\
	}
	{\end{tabular}}

\newcommand{\entry}[2]{#1:&#2\\}
\newcommand{\website}[1]{Website:&\url{#1}\\}
\newcommand{\desc}[1]{\multicolumn{2}{p{\textwidth}}{#1}\\}

\newcommand{\VV}{Verification \& Validation\xspace}
\newcommand{\vv}{verification \& validation\xspace}

\newcommand{\tbd}{\colorbox{cyan}{\%\%To Be Defined\%\%}}
\newcommand{\tbc}{\colorbox{cyan}{\%\%To Be Confirmed\%\%}}
\newcommand{\todo}[1]{\colorbox{cyan}{\%\%{#1}\%\%}}
\newcommand{\nthng}[1]{}

% The document metadata is defined below

%assign a report number here
\reportnum{OETCS/WP3/D3.6.1}

%define your workpackage here
\wp{Work-Package 3: ``Modeling''}

%set a title here
\title{D3.6.1 openETCS Functional Model}

%set a subtitle here
\subtitle{First Iteration Functional Model: ETCS Kernel Functions}

%set the date of the report here
\date{December 2014}


%document approval
%define the name and affiliation of the people involved in the documents approbation here
\creatorname{Bernd Hekele}
\creatoraffil{DB-Netz}

\techassessorname{Uwe Steinke}
\techassessoraffil{Siemens}

\qualityassessorname{Izaskun de la Torre}
\qualityassessoraffil{SQS}

\approvalname{Klaus-R\"udiger Hase}
\approvalaffil{DB Netz}


%define a list of authors and their affiliation here

\author{Bernd Hekele, Peter Mahlmann, Peyman Farhangi}

\affiliation{DB-Netz AG\\
  V\"olckerstrasse 5\\
  D-80959 M\"unchen, Germany}

\author{Uwe Steinke}

\affiliation{Siemens AG}

\author{Christian Stahl}

\affiliation{TWT-GmbH}

\author{David Mentré}
\affiliation{Mitsubishi Electric R\&D Centre Europe}

% define the coverart
\coverart[width=350pt]{openETCS_EUPL}

%define the type of report
\reporttype{Architecture and Design Specification}


\begin{abstract}
%define an abstract here
The document is used to define the deliverable of the functional model.
\end{abstract}

%=============================
\maketitle

%Modification history
%if you do not need a modification history table for your document simply comment out the eight lines below
%=============================


\chapter*{Modification History}
\tablefirsthead{
\hline 
\rowcolor{gray} 
Version & Section & Modification / Description & Author \\\hline}
\begin{supertabular}{| m{1.2cm} | m{1.2cm} | m{6.6cm} | m{4cm} |}
%\begin{supertabular}{| c | c | l | l |}
0.1 & all & Initial document providing the structure & Bernd Hekele \\\hline
0.2 & all & Added further content & Peter Mahlmann \\\hline
1.0 & --- & Version changed for submission & Peter Mahlmann \\\hline

\end{supertabular}

% list subsubsections in table of contents
\setcounter{tocdepth}{3}

\tableofcontents
%\listoffiguresandtables
\newpage
%=============================

%Uncomment the next line if you need line numbers for traceability when the document is in review
%\linenumbers
%=============================


% The actual document starts below this line
%=============================
\mainmatter

\chapter{Introduction}
This document represents deliverable D3.6.1 ``First iteration of functional model'' of the openETCS ITEA2 project and deploys the first iteration of the functional model of the ETCS onboard unit (OBU) as specified in subset-026 (system requirements specification of the ETCS OBU provided by the European Railway Agency (ERA)).

The functional model provides a semi-formal model of subset-026 and is directly linked with deliverable D3.5.1 ``First iteration of system specification model'', which focuses on the corresponding system architecture.\footnote{D3.5.1 is publicly available via the openETCS GitHub repository here: \url{https://github.com/openETCS/modeling/blob/master/deliverables/D3.5.1.pdf}}

This deliverable covers the scope of the first iteration of modeling in the openETCS project. Here, the focus was on providing kernel functions allowing a running train to read balise telegrams and determine the train position. The deliverable consists of three major parts:
\begin{itemize}
\item the Scade Model, 
\item C-code generated by Scade code generator, and
\item a textual documentation of the functional model.
\end{itemize}
The latter is also directly generated from the Scade model. The Scade model itself is located in the projects public GitHub repository and is available at:

\url{https://github.com/openETCS/modeling/releases/tag/v0.1-D3.6.1}.

The textual documentation of the functional model can be found in Appendix~\ref{a:textual} of this deliverable.

\appendix

\chapter{Textual documentation of the functional model}\label{a:textual}
The textual documentation is directly generated from the functional model via Scade. Due to the sheer size of the textual documentation, i.e.~more than 400 pages, it is not directly included in this document to keep D3.6.1 itself printable. The separate appendix can be found on the public openETCS GitHub repository:

\url{https://github.com/openETCS/modeling/blob/master/deliverables/D3.6.1.appendix.pdf}
%===================================================
%Do NOT change anything below this line

\end{document}
