%set the master document for easy compilation
%!TEX root = ../D3_5_3.tex

\paragraph{Component Requirements}

\begin{longtable}{p{.25\textwidth}p{.7\textwidth}}
\toprule
Component name			& Management\_of\_Radio\_Communication \\
\midrule
Link to SCADE model		& {\footnotesize \url{https://github.com/openETCS/modeling/tree/master/model/Scade/System/ObuFunctions/Radio/MoRC}} \\
\midrule
SCADE designer			& Uwe Steinke, Siemens \\
\midrule
Description				& 
\todo[inline]{to be checked}
The function \emph{managementOfRadioCommunication} implements the session states establishing, maintaining and termination as described in Subset-026, chap. 3.5. A SCADE state machine reflects this state model (Figure ???) accurately. Within each of the states, the activities needed as long as the state is active, are performed. When there is no communication session (state \emph{NoSession}) currently, the state machine waits for events that initiate a session (subfunction \emph{initiate\_a\_Session}). When the appropriate conditions are fulfilled, the state machine moves to the \textit{Establishing} state. Here in, it runs through the sequence required fore establishing a session (subfunction \emph{establish\_a\_Session}. Dependent on the results, the state machine changes over to the \emph{Maintaining} or \emph{Terminating} state. While in \emph{Maintaining}, the communication connection is monitored. When an event triggering the session termination occurs, the state machine switches to the state \emph{Terminating} with the subfunction \emph{terminating\_a\_CommunicationSession} and performs the session termination sequence. 

In parallel to the main state machine, \emph{managementOfRadioCommunication} monitors all the time whether the session has to be terminated (subfunction \emph{initiateTerminatingASession}) or if the session has the be terminated and subsequently established (subfunction \emph{terminateAndEstablishSession}). \emph{registeringToTheRadioNetwork} is responsible for connection to the radio network. \emph{safeRadioConnectionIndication} controls the radio connection indication for the driver.\\
\midrule
Input documents	& 
Subset-026, Chapter 3.5 \\
\midrule
Safety integrity level		& 4 \\
\midrule
Time constraints		& [If applicable description of time constraints, otherwise n/a] \\
\midrule
API requirements 		& [If applicable description of API requirements, otherwise n/a] \\
\bottomrule
\end{longtable}


\paragraph{Interface}

For an overview of the interface of this internal component we refer to the SCADE model (cf.~link above) respectively the SCADE generated documentation.