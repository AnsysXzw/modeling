\chapter{Real-time and ordering constraints}

For proper functioning of the system, the openETCS API specifies
real-time and ordering constraints that both the Application software
as well as the Basic software and remaining of the system shall
ensure. Those constraints are described in this chapter.

Some of constraints defined below are coming from Alstom Transport
(\cite{alstom-api}).

\section{System-wide constraints}

Overall, the constraints imposed on openETCS API shall ensure the ETCS
Requirements on performance are fulfilled (see \cite{subset-041} for
details and complete reference):
\begin{itemize}
\item Delay between receiving of a balise message and applying the
  emergency brake < 1 s

\item Delay between receiving of a balise message and initiating a
  communication session establishment < 1.5 s

\item Delay between receiving of a balise message and reporting the
  resulting change of status on-board < 1.5 s

\item Delay between receiving of a MA via radio (both from RBC and
  from radio in-fill) and the update of EOA on-board < 1.5 s

\item Delay between receiving of a MA from Euroloop and the update of
  EOA on-board < 1.5 s

\item Delay between receiving of a MA from Euroloop and the update of
  EOA on-board < 3 s

\item Delay between receiving of an emergency message and applying the
  reaction on-board < 1 sec, for brake order; < 1.5 sec, for
  indication to the driver

\item Delay between receiving of a radio message and initiating a
  communication session establishment < 1 sec

\item Delay between passing an EOLM and decoding of the first loop
  message $\le$ 4 s

\item Delay between driver action and new window displayed $\le$ 2s

\item Desk becomes open to ``enter Driver ID'' is displayed $\le$ 3 s

\item Desk becomes open to SH mode is displayed $\le$ 15 s

\item Delay between passing an EOA/LOA and applying the emergency
  brake < 1 s

\item Accuracy of distances measured on-board: for every measured
  distance s the accuracy shall be better or equal to $\pm$ (5m + 5\% s)

\item Accuracy of speed known on-board: $\pm$ 2 km/h for speed lower
  than 30 km/h, then increasing linearly up to $\pm$ 12 km/h at 500 km/h

\item The position of the train front indicated in a position report
  shall be estimated less than 1 sec before the beginning of sending
  of the corresponding position report

\item Safe clock drift: 0.1\%

\end{itemize}

\section{Cyclic execution of the Basic and Application software}

Both Basic and Application software shall be initialized once, in an
\define{initialization phase}, and then shall be executed cyclically
in a sequence of \define{cycles}.

During initialization phase, all units of the system shall be
initialized and be ready to proceed at the end of initialization
phase.

\section{Ordering constraints on message exchange}
\label{sec:ordering-constraints}

Between units of the system (DMI, EVC, ...), following constraints
shall be ensured:
\begin{itemize}
\item No message shall be lost \FIXME{Real meaning? Distinguish SIL0
  and SIL4?};
\item Messages sent from one unit towards another one shall be
  received in emission order;
\item Messages sent from two units towards a
single one or received by a single units from two other units shall be
received in any order.
\end{itemize}

\section{Real-time constraints}

\subsection{Event propagation delay}
\label{sec:event-propagation-delay}

The \define{external world} is the physical world out of the ETCS
system. This external world sends and receives \define{events} to/from
the ETCS system.

When an external world event is seen on a unit of the system
(e.g. balise received in BTM), this event is processed and a message
might be sent to another unit (e.g. EVC). In the reverse, a unit
(e.g. EVC) might send a message to another unit (e.g. DMI) that might
produce an event to the external world (e.g. display a message to the
driver).

The system shall ensure following constraints:
\begin{itemize}
\item The minimum delay from an input event received of the external
  world until it is received (in a message) by the Application
  software shall be \FIXME{XX ms};
\item The maximum delay from an input event of the external world
  until it is received (in a message) by the Application software
  shall be \FIXME{XX ms};
\item The minimum delay from a message sent by the Application
  software until an event is sent to the external world shall be
  \FIXME{XX ms};
\item The maximum delay from a message sent by the Application
  software until an event is sent to the external world shall be
  \FIXME{XX ms}.
\end{itemize}

\FIXME{We could impose different delays depending on considered unit.}

\subsection{Event re-ordering}

As a consequence of ordering constrains on message exchange
(§\ref{sec:ordering-constraints}) and event propagation delay
(§\ref{sec:event-propagation-delay}), two messages corresponding to
two events (e.g. reception of a balise and a radio message) on two
different units (e.g. BTM and EURORADIO) send towards a third unit
(e.g. EVC) might not be received in real time order, i.e. the message
corresponding to the first real time event might arrive in second
position on the reception unit.

\subsection{Event time-stamping}

The system is asynchronous: an external event is received on a unit and some
time is spent before seeing the corresponding message in another
unit. In order to do its computations, the Application software needs
to know at which real time the event was received. In order to do so,
a time-stamp shall be applied on all events requiring such
computations. \FIXME{Make a list of such events?}

The time-stamp shall mark the full reception of the external event at
the input unit. It shall be applied as follow:
\begin{itemize}
\item For radio-message, the time-stamp shall correspond to date of
  reception of the last byte of the message;
\item For balise, the time-stamp shall correspond to the passage of
  the train antenna over the balise center;
\item For odometry, the time-stamp shall correspond to the
  corresponding odometry event.
\end{itemize}

This time-stamp shall fulfill
following constraints:
\begin{itemize}
\item Time-stamp clock shall have a precision of 1 ms, with a
  deviation compared to real time of less than 0.1\%;
\item Time-stamp clock shall be the same on all units of the system.
\end{itemize}

\subsection{Execution time constraints}

The following real-time constrains shall be ensured:
\begin{itemize}
\item The maximum execution time taken by Application software for
  initialization step is 100 ms;
\item The maximum execution time of the Basic and Application software
  in a cycle shall be \FIXME{300 + XX ms};
\item The maximum execution time of the Application software in a
  cycle shall be 100 ms.
\end{itemize}

\section{Event burst}

In some situation, a burst of events can occur. The following
dimensioning shall be ensured:
\begin{itemize}
\item The maximum number of events sent by a unit shall be at most
  \FIXME{XX events};
\item The maximum number of events received by a unit shall be at most
  \FIXME{XX events}.
\end{itemize}

\FIXME{Do we need to define the maximum total number of events in
  fly?}

\FIXME{Do we need to define what to do when those maximums are
  reached?}

% LocalWords:  openETCS DMI EVC API ETCS Alstom BTM balise EURORADIO Euroloop
% LocalWords:  EOA LOA RBC
