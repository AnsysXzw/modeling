\section{ETCS Kernel Overview}\label{s:ETCS_Kernel_Overview}

The ETCS Kernel module consists of the 13 subcomponents F1.1 to F1.13 as depicted in Figure~\ref{f:ETCS_Kernel}. 
\begin{figure}
\center
\missingfigure{[Put SysML diagram of component here]}
\caption{F2: ETCS Kernel SysML diagram}\label{f:ETCS_Kernel}
\end{figure}
In the following we briefly describe the functionality of these subcomponents (for a more detailed description we refer to Sections~\ref{s:F2.1} to \ref{s:F2.13})
\begin{description}
\item[F2.1: Manage\_TrackSideInformation\_Integration] This component is responsible for receiving Eurobalise telegrams and Euroradio messages from the API and performs several consistency checks on the inputs.
\todo[inline]{to be checked}
\item[F2.2: Manage\_ETCS\_Procedures] This component describes the Start of Mission procedure of the train until the current status will change to another mode, level or other procedure.
\todo[inline]{descriptions needs to be improved}
\item[F2.3: trainData] Implementation of the train data with the corresponding interfaces to track, driver and RBC.
\todo[inline]{descriptions needs to be improved}
\item[F2.4: TrackAtlas] t.b.d.
\todo[inline]{to be completed}
\item[F2.5: Mode\_and\_Level] Defines the status of the ETCS
regarding on-board functional status and track infrastructure.
\todo[inline]{descriptions needs to be improved}
\item[F2.6: calculateTrainPosition] The purpose of this component is to calculate the locations of linked and unlinked balise groups (BGs) and the current train position while the train is running along the track.
\item[F2.7: SpeedSupervision\_Integration] This component monitors the speed of the train and the train location to ensure that the speed remains within the given speed and distance limits.
\todo[inline]{to be checked}
\item[F2.8: Provide\_Position\_Report] The component builds a position report for the RBC, i.e., message 132, and provides it as an output.
\todo[inline]{to be checked}
\item[F2.9: Manage\_Radio\_Communication] This component implements the onboard management of a single communication session with the track, i.e.~a single RBC. It controls the establishment, maintenance and termination process of a radio communication session and steers the underlying communication safety layer as well as the mobile device. Those and the data transfer itself are not part of this component.
\item[F2.10: ManageDMIInput] t.b.d.
\todo[inline]{to be completed}
\item[F2.11: ManageDMIOutput] t.b.d.
\todo[inline]{to be completed}
\item[F2.12: ManageTIUInput] t.b.d.
\todo[inline]{to be completed}
\item[F2.13: ManageTIUOutput] t.b.d.
\todo[inline]{to be completed}
\end{description}


\subsection{External Interfaces}
This section gives a detailed overview of the external inputs and outputs of module F2: ETCS Kernel.

\subsubsection{External Inputs}

\paragraph{[Input 1 name]}

\begin{longtable}{p{.25\textwidth}p{.7\textwidth}}
\toprule
Input name				& [Name of the input] \\
\midrule
Description				& [Brief description of the input] \\
\midrule
Source					& [Name of the source component] \\ 
\midrule
Type					& [Type of the input] \\
\midrule
Valid range of values	& [Complete list of valid values] \\
\midrule
Behaviour when value is at boundary	& [Description of components behaviour when input value is at boundary] \\
\midrule
Behaviour for values out of valid range	& [Description of components behaviour when input value is out of valid range] \\
\midrule
Behaviour when value is erroneous, absent or unwanted (i.e. spurious) & [Description of components behaviour when value is erroneous, absent or unwanted (i.e. spurious)] \\
\bottomrule
\end{longtable}

\paragraph{[Input 2 name]}

\begin{longtable}{p{.25\textwidth}p{.7\textwidth}}
\toprule
Input name				& [Name of the input] \\
\midrule
Description				& [Brief description of the input] \\
\midrule
Source					& [Name of the source component] \\ 
\midrule
Type					& [Type of the input] \\
\midrule
Valid range of values	& [Complete list of valid values] \\
\midrule
Behaviour when value is at boundary	& [Description of components behaviour when input value is at boundary] \\
\midrule
Behaviour for values out of valid range	& [Description of components behaviour when input value is out of valid range] \\
\midrule
Behaviour when value is erroneous, absent or unwanted (i.e. spurious) & [Description of components behaviour when value is erroneous, absent or unwanted (i.e. spurious)] \\
\bottomrule
\end{longtable}


\subsubsection{External Outputs}

\paragraph{[Output 1 name]}

\begin{longtable}{p{.25\textwidth}p{.7\textwidth}}
\toprule
Output name				& [Name of the output] \\
\midrule
Description				& [Brief description of the output] \\
\midrule
Destination				& [Name of the destination component(s)] \\ 
\midrule
Type					& [Type of the output] \\
\midrule
Valid range of values	& [Complete list of valid values] \\
\midrule
Behaviour when value is at boundary	& [Description of components behaviour when output value is at boundary] \\
\midrule
Behaviour for values out of valid range	& [Description of components behaviour when output value is out of valid range] \\
\midrule
Behaviour when value is erroneous, absent or unwanted (i.e. spurious) & [Description of components behaviour when value is erroneous, absent or unwanted (i.e. spurious)] \\
\bottomrule
\end{longtable}


\paragraph{[Output 2 name]}

\begin{longtable}{p{.25\textwidth}p{.7\textwidth}}
\toprule
Output name				& [Name of the output] \\
\midrule
Description				& [Brief description of the output] \\
\midrule
Destination				& [Name of the destination component(s)] \\ 
\midrule
Type					& [Type of the output] \\
\midrule
Valid range of values	& [Complete list of valid values] \\
\midrule
Behaviour when value is at boundary	& [Description of components behaviour when output value is at boundary] \\
\midrule
Behaviour for values out of valid range	& [Description of components behaviour when output value is out of valid range] \\
\midrule
Behaviour when value is erroneous, absent or unwanted (i.e. spurious) & [Description of components behaviour when value is erroneous, absent or unwanted (i.e. spurious)] \\
\bottomrule
\end{longtable}
