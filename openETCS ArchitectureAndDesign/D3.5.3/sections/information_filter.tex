%set the master document for easy compilation
%!TEX root = ../D3_5_3.tex

\paragraph{Component Requirements}
\todo[inline]{Information filter component description needs to be checked, i.e. is the description still up to date? Should more documentation provided for this component in the ADD (please check with Bernd Hekele)?}

\begin{longtable}{p{.25\textwidth}p{.7\textwidth}}
\toprule
Component name			& InformationFilter \\
\midrule
Link to SCADE model		& {\footnotesize \url{https://github.com/openETCS/modeling/tree/master/model/Scade/System/ObuFunctions/ManageLocationRelatedInformation/BaliseGroup/InformationFilter}} \\
\midrule
SCADE designer			& Christian Stahl, TWT\newline
Alexander Stante, FhG \\
\midrule
Description				& The filter receives track information (balise and radio) and filters them depending of the mode, level and source of the message. Only messages that pass the filter are valid and should be considered by other ETCS subsystems. The filter  consists of four subcomponents: FirstFilter, SecondFilter, ThirdFilter and TransitionBuffer.

\begin{description}
\item[FirstFilter] This filter performs filtering of messages
based on the current ETCS level. The decisions taken process is
described via a big decision table which contains rows for every
packet and columns for every ETCS level. This table encodes also if
certain additional information is necessary to filter a message like
pending ETCS Level transitions. Based on this filter packets of an
incoming message is either rejected, accepted or the whole message is
put in the TransitionBuffer. Messages are put in the TransitionBuffer
if there is an announced level transition and the received message is
only valid for the upcoming level.

\item[SecondFilter] The SecondFilter mainly considers messages
that are received via Euroradio. Certain messages are directly
rejected while other may be stored in the TransitionBuffer. The buffer
is used to store messages that are received from non supervising RBCs,
but will be reevaluated after a RBC transition.

\item[ThirdFilter] The last filter is functionally very similar
the the FirstFilter, however it filters depending on the mode. It also
contains a decision table with rows for every packet but the columns
are modes.

\item[TransitionBuffer] The InformationFilter uses two
TransitionBuffers. One is used to store up to three messages for the
ETCS level transition and the other buffer is used for RBC
transitions. The buffer is designed as a ring buffer and message are
read in FIFO order.
\end{description} \\
\midrule
Input documents	& 
  Subset-026, Chapter 4.8 \\
\midrule
Safety integrity level	& 4 \\
\midrule
Time constraints		& n/a \\
\midrule
API requirements 		& n/a \\
\bottomrule
\end{longtable}


\paragraph{Interface}

For an overview of the interface of this internal component we refer to the SCADE model (cf.~link above) respectively the SCADE generated documentation.