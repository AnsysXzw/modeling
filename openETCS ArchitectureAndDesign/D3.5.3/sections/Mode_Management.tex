%set the master document for easy compilation
%!TEX root = ../D3_5_2.tex

\paragraph{Component Requirements}

\begin{longtable}{p{.25\textwidth}p{.7\textwidth}}
\toprule
Component name			& Mode\_Management \\
\midrule
Link to SCADE model		& {\footnotesize \url{https://github.com/openETCS/modeling/tree/master/model/Scade/System/ObuFunctions/ManageLevelsAndModes/Modes}} \\
\midrule
SCADE designer			& Marielle Petit-Doche, Systerel \\
\midrule
Description				& This function is in charge of the computation of new mode to apply according to conditions from inputs (track information, driver interactions, train data,...) and other functions.

Three subfunctions are defined:
\begin{description}
\item[Inputs] proceeds to inputs check and preparation.
\item[ComputeModesCondition] performs all specific procedure linked to mode management and defined in  \citep{subset-026} sections 5.4, 5.5, 5.6, 5.7, 5.8, 5.9, 5.11, 5.12, 5.13, 5.19 and specifies the conditions to define a mode transition according condition table of section 4.6.3 of \citep{subset-026}
\item[SwitchModes] performs the mode selection according the conditions and priorities defined in transition table  section 4.6.2 of \citep{subset-026}
\item[Outputs] prepares packet of outputs.
\end{description} \\
\midrule
Input documents	& 
Subset-026, Chapter 4.4, 4.6, 5.4, 5.5, 5.6, 5.7, 5.8, 5.9, 5.11, 5.12, 5.13, 5.19 \\
\midrule
Safety integrity level		& 4 \\
\midrule
Time constraints		& [If applicable description of time constraints, otherwise n/a] \\
\midrule
API requirements 		& [If applicable description of API requirements, otherwise n/a] \\
\bottomrule
\end{longtable}


\paragraph{Interface}

For an overview of the interface of this internal component we refer to the SCADE model (c.f.~link above) respectively the SCADE generated documentation.