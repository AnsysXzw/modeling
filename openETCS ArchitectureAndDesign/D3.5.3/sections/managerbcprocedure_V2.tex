%set the master document for easy compilation
%!TEX root = ../D3_5_3.tex

\section{F2.9: Manage\_Radio\_Communication and RBC\_Handover}
\todo[inline]{section has to be completed. The level of this component in final architecture has been fixed.}
\subsection{Component Requirements}

\begin{longtable}{p{.25\textwidth}p{.7\textwidth}}
\toprule
Component name			& MoRC\_HO \\
\midrule
Link to SCADE model		& {\footnotesize \url{???}} \\
\midrule
SCADE designer			& Uwe Steinke, Siemens AG \\
\midrule
Description				&  
\todo[inline]{to be checked}
The management of radio communication MoRC implements the onboard management part of a single communication session with the track, i.e. a single RBC. It controls the establishing, maintaining and termination process of a radio communication session and steers the underlying communication safety layer and the mobile device. Those and the data transfer itself are not part of the function. 

The kernel function of the MoRC component is \emph{managementOfRadioCommunication} (figure ???). The implementation is kept close to the prose of Subset-026, chap. 3.5. Since chap. 3.5 rarely refers to terms, variable types, packets and messages of the ETCS language as specified in Subset-026, chap. 7 and 8, \emph{managementOfRadioCommunication} does neither. 

To be capable of being integrated with other OBU software components, \emph{MoRC} had to be wrapped with a transformer between the ETCS and the "chap. 3.5" language. This is the purpose of the main function of \emph{MoRC}, \emph{MoRC\_Main}.  \\
\midrule
Input documents	& 
Subset-026, Chapter 4 \newline
Subset-026, Chapter 5 \\
\midrule
Safety integrity level	& 4 \\
\midrule
Time constraints		& [If applicable description of time constraints, otherwise n/a] \\
\midrule
API requirements 		& [If applicable description of API requirements, otherwise n/a] \\
\bottomrule
\end{longtable}


\subsection{Interface}

An overview of the interface of component Manage\_Radio\_Communication is shown in Figure~\ref{f:manage_radio_communication_interface}. The inputs and outputs are described in detail in Section~\ref{s:manage_radio_communication_inputs} respectively \ref{s:manage_radio_communication_outputs}. Sub components are described in Section~\ref{s:manage_radio_communication_subcomponents}.

\begin{figure}
\center
\missingfigure{[Put SysML diagram of component here]}
\caption{Manage\_Radio\_Communication component SysML diagram}\label{f:manage_radio_communication_interface}
\end{figure}


\subsubsection{Inputs}\label{s:manage_radio_communication_inputs}

\paragraph{[Input 1 name]}

\begin{longtable}{p{.25\textwidth}p{.7\textwidth}}
\toprule
Input name				& [Name of the input] \\
\midrule
Description				& [Brief description of the input] \\
\midrule
Source					& [Name of the source component] \\ 
\midrule
Type					& [Type of the input] \\
\midrule
Valid range of values	& [Complete list of valid values] \\
\midrule
Behaviour when value is at boundary	& [Description of components behaviour when input value is at boundary] \\
\midrule
Behaviour for values out of valid range	& [Description of components behaviour when input value is out of valid range] \\
\midrule
Behaviour when value is erroneous, absent or unwanted (i.e. spurious) & [Description of components behaviour when value is erroneous, absent or unwanted (i.e. spurious)] \\
\bottomrule
\end{longtable}


\paragraph{[Input 2 name]}

\begin{longtable}{p{.25\textwidth}p{.7\textwidth}}
\toprule
Input name				& [Name of the input] \\
\midrule
Description				& [Brief description of the input] \\
\midrule
Source					& [Name of the source component] \\ 
\midrule
Type					& [Type of the input] \\
\midrule
Valid range of values	& [Complete list of valid values] \\
\midrule
Behaviour when value is at boundary	& [Description of components behaviour when input value is at boundary] \\
\midrule
Behaviour for values out of valid range	& [Description of components behaviour when input value is out of valid range] \\
\midrule
Behaviour when value is erroneous, absent or unwanted (i.e. spurious) & [Description of components behaviour when value is erroneous, absent or unwanted (i.e. spurious)] \\
\bottomrule
\end{longtable}


\subsubsection{Outputs}\label{s:manage_radio_communication_outputs}

\paragraph{[Output 1 name]}

\begin{longtable}{p{.25\textwidth}p{.7\textwidth}}
\toprule
Output name				& [Name of the output] \\
\midrule
Description				& [Brief description of the output] \\
\midrule
Destination				& [Name of the destination component(s)] \\ 
\midrule
Type					& [Type of the output] \\
\midrule
Valid range of values	& [Complete list of valid values] \\
\midrule
Behaviour when value is at boundary	& [Description of components behaviour when output value is at boundary] \\
\midrule
Behaviour for values out of valid range	& [Description of components behaviour when output value is out of valid range] \\
\midrule
Behaviour when value is erroneous, absent or unwanted (i.e. spurious) & [Description of components behaviour when value is erroneous, absent or unwanted (i.e. spurious)] \\
\bottomrule
\end{longtable}


\paragraph{[Output 2 name]}

\begin{longtable}{p{.25\textwidth}p{.7\textwidth}}
\toprule
Output name				& [Name of the output] \\
\midrule
Description				& [Brief description of the output] \\
\midrule
Destination				& [Name of the destination component(s)] \\ 
\midrule
Type					& [Type of the output] \\
\midrule
Valid range of values	& [Complete list of valid values] \\
\midrule
Behaviour when value is at boundary	& [Description of components behaviour when output value is at boundary] \\
\midrule
Behaviour for values out of valid range	& [Description of components behaviour when output value is out of valid range] \\
\midrule
Behaviour when value is erroneous, absent or unwanted (i.e. spurious) & [Description of components behaviour when value is erroneous, absent or unwanted (i.e. spurious)] \\
\bottomrule
\end{longtable}


\subsection{Subcomponents}\label{s:manage_radio_communication_subcomponents}

\todo[inline]{Maybe more subcomponents have to be added. Please check this.}

\subsubsection{Management\_of\_Radio\_Communication}
%set the master document for easy compilation
%!TEX root = ../D3_5_3.tex

\paragraph{Component Requirements}

\begin{longtable}{p{.25\textwidth}p{.7\textwidth}}
\toprule
Component name			& MoRC\_Main\_v2 (Management\_of\_Radio\_Communication) \\
\midrule
Link to SCADE model		& {\footnotesize \url{https://github.com/openETCS/modeling/tree/master/model/Scade/System/ObuFunctions/Radio/MoRC}} \\
\midrule
SCADE designer			& Uwe Steinke, Siemens \\
\midrule
Description				& 
The function \emph{MoRC\_Main\_v2} implements the session states establishing, maintaining and terminating as described in Subset-026, chap. 3.5. A SCADE state machine reflects this state model  accurately. Within each of the states, the activities needed as long as the state is active, are performed. \newline

\emph{MoRC\_Main\_v2} is related to exactly one of the radio mobile modems onboard, monitors its status and controls the processes of registration to the radio network, connecting to one RBC and establishing a radio session with the RBC. \emph{MoRC\_Main\_v2} communicates with its mobile modem directly via the API.  \newline

As the OBU is required to manage up to two RBCs,  two instances of \emph{MoRC\_Main\_v2} are used.  \newline

In addition, \emph{MoRC\_Main\_v2} generates the radio connection indication for the driver.

\\
\midrule
Input documents	& 
Subset-026, Chapter 3.5 \\
\midrule
Safety integrity level		& 4 \\
\midrule
Time constraints		& Implements several time delays, therefore appropriate clocking required \\
\midrule
API requirements 		& Interfaces to the OBUs mobile modem hardware via API \\
\bottomrule
\end{longtable}


\paragraph{Interface}

For an overview of the interface of this internal component we refer to the SCADE model (cf.~link above) respectively the SCADE generated documentation.

\subsubsection{genMSGToRBC}
%set the master document for easy compilation
%!TEX root = ../D3_5_3.tex

\paragraph{Component Requirements}

\begin{longtable}{p{.25\textwidth}p{.7\textwidth}}
\toprule
Component name			& genMSGToRBC \\
\midrule
Link to SCADE model		& {\footnotesize \url{https://github.com/openETCS/modeling/tree/master/model/Scade/System/ObuFunctions/Radio/MoRC}} \\
\midrule
SCADE designer			& Uwe Steinke, Siemens \\
\midrule
Description				& 
\todo[inline]{to be completed}\\
\midrule
Input documents	& 
Subset-026, Chapter 3.5 \\
\midrule
Safety integrity level	& 4 \\
\midrule
Time constraints		& [If applicable description of time constraints, otherwise n/a] \\
\midrule
API requirements 		& [If applicable description of API requirements, otherwise n/a] \\
\bottomrule
\end{longtable}


\paragraph{Interface}

For an overview of the interface of this internal component we refer to the SCADE model (cf.~link above) respectively the SCADE generated documentation.
