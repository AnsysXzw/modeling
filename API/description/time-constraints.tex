\chapter{Real-time and ordering constraints}

For proper functioning of the system, the openETCS API specifies
real-time and ordering constraints that both the Application software
as well as the Basic software and remaining of the system shall
ensure. Those constraints are described in this chapter.

Overall, those constraints shall ensure the ETCS Requirements on
performance (\cite{subset-041}) are fulfilled. Some of those
constraints are coming from Alstom Transport (\cite{alstom-api}).

\section{Cyclic execution of the Basic and Application software}

Both Basic and Application software shall be initialized once, in an
\define{initialization phase}, and then shall be executed cyclically
in a sequence of \define{cycles}.

During initialization phase, all units of the system shall be
initialized and be ready to proceed at the end of initialization
phase.

\section{Ordering constraints on message exchange}
\label{sec:ordering-constraints}

Between units of the system (DMI, EVC, ...), following constraints
shall be ensured:
\begin{itemize}
\item No message shall be lost;
\item Messages sent from one unit towards another one shall be
  received in emission order;
\item Messages sent from two units towards a
single one or received by a single units from two other units shall be
received in any order.
\end{itemize}

\section{Real-time constraints}

\subsection{Event propagation delay}
\label{sec:event-propagation-delay}

The \define{external world} is the physical world out of the ETCS
system. This external world sends and receives \define{events} to/from
the ETCS system.

When an external world event is seen on a unit of the system
(e.g. balise received in BTM), this event is processed and a message
might be sent to another unit (e.g. EVC). In the reverse, a unit
(e.g. EVC) might send a message to another unit (e.g. DMI) that might
produce an event to the external world (e.g. display a message to the
driver).

The system shall ensure following constraints:
\begin{itemize}
\item The minimum delay from an input event received of the external
  world until it is received (in a message) by the Application
  software shall be \FIXME{XX ms};
\item The maximum delay from an input event of the external world
  until it is received (in a message) by the Application software
  shall be \FIXME{XX ms};
\item The minimum delay from a message sent by the Application
  software until an event is sent to the external world shall be
  \FIXME{XX ms};
\item The maximum delay from a message sent by the Application
  software until an event is sent to the external world shall be
  \FIXME{XX ms}.
\end{itemize}

\subsection{Event re-ordering}

As a consequence of ordering constrains on message exchange
(§\ref{sec:ordering-constraints}) and event propagation delay
(§\ref{sec:event-propagation-delay}), two messages corresponding to
two events (e.g. reception of a balise and a radio message) on two
different units (e.g. BTM and EURORADIO) send towards a third unit
(e.g. EVC) might not be received in real time order, i.e. the message
corresponding to the first real time event might arrive in second
position on the reception unit.

\subsection{Event time-stamping}

The system is asynchronous: an event is received on a unit and some
time is spent before seeing the corresponding message in another
unit. In order to do its computations, the Application software needs
to know at which real time the event was received. In order to do so,
a time-stamp shall be applied on all events requiring such
computations. \FIXME{Make a list of such events?}

This time-stamp shall fulfill
following constraints:
\begin{itemize}
\item Time-stamp clock shall have a precision of 1 µs, with a
  deviation compared to real time of less than 0.1\%;
\item Time-stamp clock shall be the same on all units of the system.
\end{itemize}

\subsection{Execution time constraints}

The following real-time constrains shall be ensured:
\begin{itemize}
\item The maximum execution time taken by Application software for
  initialization step is 100 ms;
\item The maximum execution time of the Basic and Application software
  in a cycle shall be \FIXME{300 + XX ms};
\item The maximum execution time of the Application software in a
  cycle shall be 100 ms.
\end{itemize}

\section{Event burst}

In some situation, a burst of events can occur. The following
dimensioning shall be ensured:
\begin{itemize}
\item The maximum number of event sent by a unit shall be at most
  \FIXME{XX events};
\item The maximum number of event received by a unit shall be at most
  \FIXME{XX events}.
\end{itemize}

\FIXME{Do we need to define the total number of events?}

\FIXME{Do we need to define what to do when those maximums are
  reached?}

% LocalWords:  openETCS DMI EVC API ETCS Alstom BTM balise EURORADIO
