
\subsection{Motivation}
\label{sec:Motivation}

The openETCS work package WP3 aims to provide the kernel architecture and the design of the openETCS OBU software as mainly specified in UNISIG Subset\_026 version\_3.3.0. 

The appropriate functionality has been divided into a list of functions of different complexity (see \url{https://github.com/openETCS/SRS-Analysis/blob/master/System Analysis/List_Functions.xlsx}).

All these functions are object of the openETCS project and have to be analysed from their requirements and subsequently modelled and implemented. With limited manpower, a reasonable selection and order of these functions is required for the practical work that allows the distribution of the workload, more openETCS participants to join and leads to an executable---limited---kernel function as soon as possible. 

While the first version of this document focuses on the first version of the limited kernel function, it is intended to grow in parallel to the growing openETCS software.


\subsection{Objectives}
\label{sec:Objectives}



The first objective of WP3 software shall be
\begin{itemize}
	\item ``Make the train run as soon as possible, with a very minimum functionality, and in the form of a rapid prototype.''
\end{itemize}
This does not contradict the openETCS goal to conform to EN50128.
\begin{itemize}
	\item After a phase of prototyping, the openETCS software shall be implemented in compliance to EN50128 for SIL4 systems.
\end{itemize}
Additional goals for this document are
\begin{itemize}
	\item Identification of the functions required for a minimum OBU kernel
	\item Architecture overview regarding the minimum OBU kernel
	\item Technical approach: Description of the proceeding and methods to be used
	\item Road map of the minimum OBU kernel functions
	\item Road map thereafter
\end{itemize}

Note: This document will be extended according to the progress of WP3. 



