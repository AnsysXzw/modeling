%set the master document for easy compilation
%!TEX root = ../D3_5_2.tex

\section{Component Requirements}

\begin{longtable}{p{.25\textwidth}p{.7\textwidth}}
\toprule
Component name			& [Component name] \\
\midrule
Link to SCADE model		& {\footnotesize \url{https://github.com/openETCS/modeling/tree/master/model/Scade/System/ObuFunctions/ManageLocationRelatedInformation/BaliseGroup/Receive_TrackSide_Msg}} \\
\midrule
SCADE designer			& [Name, affiliation] \\
\midrule
Description				& [Brief description of the components functionality] \\
\midrule
Input documents	& 
Subset-026, version, chapter ?.?\newline
Subset-026, version, chapter ?.?\newline
Subset-026, version, chapter ?.?.?\\
\midrule
Safety integrity level		& 4 \\
\midrule
Time constraints		& [If applicable description of time constraints, otherwise n/a] \\
\midrule
API requirements 		& [If applicable description of API requirements, otherwise n/a] \\
\bottomrule
\end{longtable}


\section{Interface}

An overview of the interface of component [component name] is shown in Figure~\ref{f:template_interface}. The inputs and outputs are described in detail in Section~\ref{s:template_inputs} respectively \ref{s:template_outputs}.

\begin{figure}
\center
{[Put SysML diagram of component here]}
\caption{Component SysML diagram}\label{f:template_interface}
\end{figure}


\subsection{Inputs}\label{s:template_inputs}

\subsubsection{[Input 1 name]}

\begin{longtable}{p{.25\textwidth}p{.7\textwidth}}
\toprule
Input name				& [Name of the input] \\
\midrule
Description				& [Brief description of the input] \\
\midrule
Source					& [Name of the source component] \\ 
\midrule
Type					& [Type of the input] \\
\midrule
Valid range of values	& [Complete list of valid values] \\
\midrule
Behaviour when value is at boundary	& [Description of components behaviour when input value is at boundary] \\
\midrule
Behaviour for values out of valid range	& [Description of components behaviour when input value is out of valid range] \\
\bottomrule
\end{longtable}


\subsubsection{[Input 2 name]}

\begin{longtable}{p{.25\textwidth}p{.7\textwidth}}
\toprule
Input name				& [Name of the input] \\
\midrule
Description				& [Brief description of the input] \\
\midrule
Source					& [Name of the source component] \\ 
\midrule
Type					& [Type of the input] \\
\midrule
Valid range of values	& [Complete list of valid values] \\
\midrule
Behaviour when value is at boundary	& [Description of components behaviour when input value is at boundary] \\
\midrule
Behaviour for values out of valid range	& [Description of components behaviour when input value is out of valid range] \\
\bottomrule
\end{longtable}


\subsection{Outputs}\label{s:template_outputs}

\subsubsection{[Output 1 name]}

\begin{longtable}{p{.25\textwidth}p{.7\textwidth}}
\toprule
Output name				& [Name of the output] \\
\midrule
Description				& [Brief description of the output] \\
\midrule
Destination				& [Name of the destination component(s)] \\ 
\midrule
Type					& [Type of the output] \\
\midrule
Valid range of values	& [Complete list of valid values] \\
\midrule
Behaviour when value is at boundary	& [Description of components behaviour when output value is at boundary] \\
\midrule
Behaviour for values out of valid range	& [Description of components behaviour when output value is out of valid range] \\
\bottomrule
\end{longtable}


\subsubsection{[Output 2 name]}

\begin{longtable}{p{.25\textwidth}p{.7\textwidth}}
\toprule
Output name				& [Name of the output] \\
\midrule
Description				& [Brief description of the output] \\
\midrule
Destination				& [Name of the destination component(s)] \\ 
\midrule
Type					& [Type of the output] \\
\midrule
Valid range of values	& [Complete list of valid values] \\
\midrule
Behaviour when value is at boundary	& [Description of components behaviour when output value is at boundary] \\
\midrule
Behaviour for values out of valid range	& [Description of components behaviour when output value is out of valid range] \\
\bottomrule
\end{longtable}
