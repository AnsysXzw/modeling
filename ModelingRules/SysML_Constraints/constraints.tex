\documentclass{template/openetcs_article}
% Use the option "nocc" if the document is not licensed under Creative Commons
%\documentclass[nocc]{template/openetcs_article}
\usepackage{lipsum,url}
\usepackage{supertabular}
\usepackage{multirow}
\usepackage{color, colortbl}
\definecolor{gray}{rgb}{0.8,0.8,0.8}
\usepackage[modulo]{lineno}
\graphicspath{{./template/}{.}{./images/}}
\begin{document}
\frontmatter
\project{openETCS}

%Please do not change anything above this line
%============================



% The document metadata is defined below

%assign a report number here
\reportnum{OETCS/TODO}

%define your workpackage here
\wp{Work-Package 3: ``Modeling''}

%set a title here
\title{Definition of Constraints for Modeling in SysML}

%set a subtitle here
\subtitle{A template for short document. Adapted from report template.}

%set the date of the report here
\date{May 2014}

%define a list of authors and their affiliation here

\author{Alexander Stante}

\affiliation{Fraunhofer Institute for Embedded Systems \\
  and Communication Technologies ESK\\
  Hansastr. 32\\
  80686 Muenchen, Germany}

% define the coverart
\coverart[width=350pt]{openETCS_EUPL}

%define the type of report
\reporttype{Description of work}


\begin{abstract}
%define an abstract here
  \lipsum[12-13]
\end{abstract}

%=============================
\maketitle

%Modification history
%if you do not need a modification history table for your document simply comment out the eight lines below
%=============================
\section*{Modification History}
\tablefirsthead{
\hline 
\rowcolor{gray} 
Version & Section & Modification / Description & Author \\\hline}
\begin{supertabular}{| m{1.2cm} | m{1.2cm} | m{6.6cm} | m{4cm} |}
 & & & \\\hline
\end{supertabular}


\tableofcontents
\listoffiguresandtables
\newpage
%=============================

%Uncomment the next line if you need line numbers for tracebility when the document is in review
%\linenumbers
%=============================


% The actual document starts below this line
%=============================

\section{Introduction}

\section{Constraints}
\label{sec:constraints}

\subsection{Modelng Guidelines}
\label{sec:modeling-guidelines}

\begin{longtable}{|l||>{\raggedright}p{0.85\linewidth}|}
  \hline
  \textbf{Name}        &  Ports have an associated type \tabularnewline \hline
  \textbf{Target}      &  \texttt{uml.Port} \tabularnewline \hline
  \textbf{Stereotype}  &  n/a \tabularnewline \hline
  \textbf{Severity}    &  Error \tabularnewline \hline
  \textbf{Mode}        &  Batch \tabularnewline \hline
  \textbf{Message}     &  The Port '\texttt{[name]}' has no associated type. Please add a 
                          type to the port. \tabularnewline \hline
  \textbf{Rationale}   &  Ports define an interface between different SysML blocks. The type is a necessary 
                          part of the port definition to completely specify the interface and to allow code 
                          generation and V\&V of the SysML model. \tabularnewline \hline
  \textbf{Description} &  For every UML Port check if a associated type is set, i.e. the type is not 
                          \texttt{null}. \tabularnewline \hline
\end{longtable}



\subsection{Tool Constraints}
\label{sec:tool-constraints}

\subsubsection{C99 Code Generation}

\begin{longtable}{|l||>{\raggedright}p{0.85\linewidth}|}
  \hline
  \textbf{Name}        & Name is not a reserved C99 keyword \tabularnewline \hline
  \textbf{Target}      & \texttt{uml.NamedElement} \tabularnewline \hline
  \textbf{Stereotype}  & n/a \tabularnewline \hline
  \textbf{Severity}    & Error \tabularnewline \hline
  \textbf{Mode}        & Batch \tabularnewline \hline
  \textbf{Message}     & The used name '\texttt{[name]}' is a reserved C99 keyword in ISO/IEC 9899:1999. 
                         Please change the name of the element. \tabularnewline \hline
  \textbf{Rationale}   & If the SysML model is used to generate C code (for example through Scade), 
                         a straight-forward implementation uses the name of NamedElements  
                         for the generated C data structures, functions, etc. Using reserved keywords 
                         for variables, functions, etc. would result in  compilation errors. Therefore, 
                         all NamedElements are restricted to not contain reserved C99 keywords according
                         to ISO/IEC 9899:1999. \tabularnewline \hline
  \textbf{Description} & The name of each NamedElement is checked against a list of C99 keywords which 
                         are defined in ISO/IEC 9899:1999 section ``6.4.1 Keywords''. The keywords are: 
                         \begin{tabular}{l l l}
                           \texttt{auto}     & \texttt{if}       & \texttt{unsigned}       \\
                           \texttt{break}    & \texttt{inline}   & \texttt{void}           \\
                           \texttt{case}     & \texttt{int}      & \texttt{volatile}       \\
                           \texttt{char}     & \texttt{long}     & \texttt{while}          \\
                           \texttt{const}    & \texttt{register} & \texttt{\_Alignas}       \\
                           \texttt{continue} & \texttt{restrict} & \texttt{\_Alignof}       \\
                           \texttt{default}  & \texttt{return}   & \texttt{\_Atomic}        \\
                           \texttt{do}       & \texttt{short}    & \texttt{\_Bool}          \\
                           \texttt{double}   & \texttt{signed}   & \texttt{\_Complex}       \\
                           \texttt{else}     & \texttt{sizeof}   & \texttt{\_Generic}       \\
                           \texttt{enum}     & \texttt{static}   & \texttt{\_Imaginary}     \\
                           \texttt{extern}   & \texttt{struct}   & \texttt{\_Noreturn}      \\
                           \texttt{float}    & \texttt{switch}   & \texttt{\_Static\_assert} \\
                           \texttt{for}      & \texttt{typedef}  & \texttt{\_Thread\_local}  \\
                           \texttt{goto}     & \texttt{union} \\
                         \end{tabular}\tabularnewline \hline
\end{longtable}

\begin{longtable}{|l||>{\raggedright}p{0.85\linewidth}|}
  \hline
  \textbf{Name}        &  Name is valid C99 identifier \tabularnewline \hline
  \textbf{Target}      &  \texttt{uml.NamedElement} \tabularnewline \hline
  \textbf{Stereotype}  &  n/a \tabularnewline \hline
  \textbf{Severity}    &  Error \tabularnewline \hline
  \textbf{Mode}        &  Batch \tabularnewline \hline
  \textbf{Message}     &  The used name '\texttt{[name]}' is not a valid C99 identifier as specified in 
                          ISO/IEC 9899:1999. Please change the name of the element. \tabularnewline \hline
  \textbf{Rationale}   &  Same rational as for previous constraint \tabularnewline \hline
  \textbf{Description} &  The name of each NamedElement is checked against the identifier specification
                          in ISO/IEC 9899:1999 section ``6.4.2 Identifiers''. The sepecification equals to the
                          following regular expression: \texttt{[\_a-zA-Z][0-9a-zA-Z\*]}. \tabularnewline \hline
\end{longtable}


\subsubsection{Classical B Generation}

\subsubsection{POSIX Platform Generation}

\begin{longtable}{|l||>{\raggedright}p{0.85\linewidth}|}
  \hline
  \textbf{Name}        &  Name does not conflict with POSIX.1-2008 reserved names \tabularnewline \hline
  \textbf{Target}      &  \texttt{uml.NamedElement} \tabularnewline \hline
  \textbf{Stereotype}  &  n/a \tabularnewline \hline
  \textbf{Severity}    &  Error \tabularnewline \hline
  \textbf{Mode}        &  Batch \tabularnewline \hline
  \textbf{Message}     &  The name '\texttt{[name]}' ends with '\texttt{\_t}' which is reserved for
                          implementors and a POSIX.1-2008 conforming application shall avoid. 
                          Please change the name of the element. \tabularnewline \hline
  \textbf{Rationale}   &  Implementors of POSIX.1-2008 can introduce new types. To avoid clashes with this types,
                          POSIX.1-2008 requires that conforming applications avoid names which 
                          ends with '\texttt{\_t}'. \tabularnewline \hline
  \textbf{Description} &  The name of each NamedElement is checked if its ends with '\texttt{\_t}'. This is equal to 
                          the following regular expression: \texttt{![.*\_t]}. \tabularnewline \hline
\end{longtable}



% \bibliographystyle{unsrt}
% \bibliography{erdc}

%===================================================
%Do NOT change anything below this line

\end{document}
