%set the master document for easy compilation
%!TEX root = ../D3_5_3.tex

\section{F2.10: manageDMI\_input}\label{s:F2.10}



\subsection{Component Requirements}

\begin{longtable}{p{.25\textwidth}p{.7\textwidth}}
\toprule
Component name			& manageDMI\_input \\
\midrule
Link to SCADE model		& {\footnotesize \url{https://github.com/openETCS/modeling/tree/master/model/Scade/System/ObuFunctions/manageData}} \\
\midrule
SCADE designer			& Bernd Hekele, DB Netz AG \\
\midrule
Description				&in this module, the incoming messages from the Driver Machine Interface (DMI) will be processed and provided.
The point at which the driver needs to start braking to avoid intervention by the ETCS onboard equipment.\\
\midrule
Input documents	& 
ERA ERTMS 015560\newline
ETCS DRIVER MACHINE INTERFACE\\
\midrule
Safety integrity level		& 4 \\
\midrule
Time constraints		& \todo[inline]{section and corresponding subsections have to be completed} \\
\midrule
API requirements 		&\todo[inline]{section and corresponding subsections have to be completed} \\
\bottomrule
\end{longtable}


\subsection{Interface}

An overview of the interface of component manageDMI\_input is shown in Figure~\ref{f:ManageDMIInput}. The inputs and outputs are described in detail in Section~\ref{s:ManageDMIInput_inputs} respectively \ref{s:ManageDMIInput_outputs}. Subcomponents are described in Section~\ref{s:ManageDMIInput_subcomponents}.

\begin{figure}
\center
\includegraphics[width=\textwidth]{images/F2_10_manageDMI_input.pdf}
\caption{manageDMI\_input SysML diagram.}\label{f:ManageDMIInput}
\end{figure}


\subsubsection{Inputs}\label{s:ManageDMIInput_inputs}

\paragraph{[Input 1 name]}

\begin{longtable}{p{.25\textwidth}p{.7\textwidth}}
\toprule
Input name				& [Name of the input] \\
\midrule
Description				& [Brief description of the input] \\
\midrule
Source					& [Name of the source component] \\ 
\midrule
Type					& [Type of the input] \\
\midrule
Valid range of values	& [Complete list of valid values] \\
\midrule
Behaviour when value is at boundary	& [Description of components behaviour when input value is at boundary] \\
\midrule
Behaviour for values out of valid range	& [Description of components behaviour when input value is out of valid range] \\
\midrule
Behaviour when value is erroneous, absent or unwanted (i.e. spurious) & [Description of components behaviour when value is erroneous, absent or unwanted (i.e. spurious)] \\
\bottomrule
\end{longtable}


\paragraph{[Input 2 name]}

\begin{longtable}{p{.25\textwidth}p{.7\textwidth}}
\toprule
Input name				& [Name of the input] \\
\midrule
Description				& [Brief description of the input] \\
\midrule
Source					& [Name of the source component] \\ 
\midrule
Type					& [Type of the input] \\
\midrule
Valid range of values	& [Complete list of valid values] \\
\midrule
Behaviour when value is at boundary	& [Description of components behaviour when input value is at boundary] \\
\midrule
Behaviour for values out of valid range	& [Description of components behaviour when input value is out of valid range] \\
\midrule
Behaviour when value is erroneous, absent or unwanted (i.e. spurious) & [Description of components behaviour when value is erroneous, absent or unwanted (i.e. spurious)] \\
\bottomrule
\end{longtable}


\subsubsection{Outputs}\label{s:ManageDMIInput_outputs}

\paragraph{[Output 1 name]}

\begin{longtable}{p{.25\textwidth}p{.7\textwidth}}
\toprule
Output name				& [Name of the output] \\
\midrule
Description				& [Brief description of the output] \\
\midrule
Destination				& [Name of the destination component(s)] \\ 
\midrule
Type					& [Type of the output] \\
\midrule
Valid range of values	& [Complete list of valid values] \\
\midrule
Behaviour when value is at boundary	& [Description of components behaviour when output value is at boundary] \\
\midrule
Behaviour for values out of valid range	& [Description of components behaviour when output value is out of valid range] \\
\midrule
Behaviour when value is erroneous, absent or unwanted (i.e. spurious) & [Description of components behaviour when value is erroneous, absent or unwanted (i.e. spurious)] \\
\bottomrule
\end{longtable}


\paragraph{[Output 2 name]}

\begin{longtable}{p{.25\textwidth}p{.7\textwidth}}
\toprule
Output name				& [Name of the output] \\
\midrule
Description				& [Brief description of the output] \\
\midrule
Destination				& [Name of the destination component(s)] \\ 
\midrule
Type					& [Type of the output] \\
\midrule
Valid range of values	& [Complete list of valid values] \\
\midrule
Behaviour when value is at boundary	& [Description of components behaviour when output value is at boundary] \\
\midrule
Behaviour for values out of valid range	& [Description of components behaviour when output value is out of valid range] \\
\midrule
Behaviour when value is erroneous, absent or unwanted (i.e. spurious) & [Description of components behaviour when value is erroneous, absent or unwanted (i.e. spurious)] \\
\bottomrule
\end{longtable}


\subsection{Subcomponents}\label{s:ManageDMIInput_subcomponents}

Currently ManageDMIInput does not have any subcomponents.

%\subsubsection{ManageTextMessages}
%%set the master document for easy compilation
%!TEX root = ../D3_5_3.tex

\paragraph{Component Requirements}

\begin{longtable}{p{.25\textwidth}p{.7\textwidth}}
\toprule
Component name			& ManageTextMessages \\
\midrule
Link to SCADE model		& {\footnotesize \url{https://github.com/openETCS/modeling/tree/master/model/Scade/System/ObuFunctions/manageData/manageDMI}} \\
\midrule
SCADE designer			& Bernd Hekele, DB Netz AG \\
\midrule
Description				& This subcomponent receives available text messages from within the EVC sources, handles messages according to the priority, and provides an output stack for messages. \\
\midrule
Input documents			& 
ERA ERTMS 015560\newline
ETCS DRIVER MACHINE INTERFACE\newline
ERSA API\\
\midrule
Safety integrity level	& 4 \\
\midrule
Time constraints		&  n/a \\
\midrule
API requirements 		&  n/a \\
\bottomrule
\end{longtable}


\paragraph{Interface}

For an overview of the interface of this internal component we refer to the SCADE model (cf.~link above) respectively the SCADE generated documentation.

