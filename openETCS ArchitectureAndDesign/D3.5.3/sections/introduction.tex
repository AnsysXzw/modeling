%set the master document for easy compilation
%!TEX root = ../D3_5_3.tex

\chapter{Introduction}

A primary goal of the openETCS ITEA2 project is to provide a formal specification and a non-vital reference implementation of an ETCS onboard unit (OBU) according to the specification described in Subset-026 \cite{subset-026}, defined the European Railway Agency (ERA). 

This deliverable, i.e.~D3.5.x, describes the architecture and design specification of the openETCS onboard (OBU) model. As the development of the OBU model is done iteratively according to a SCRUM process, the last digit of the deliverable identifier, i.e.~x, denotes the current iteration of the model. This document should be considered as a complement to the following project outcomes respectively deliverables:
\begin{itemize}
\item the corresponding SysML and SCADE models, available at \url{https://github.com/openETCS/modeling/tree/master/model/Scade/System},
\item the corresponding functional design description, i.e.~D3.6.x, and
\item the documentation of the generic openETCS Application Programming Interface (API), available at \url{https://github.com/openETCS/modeling/blob/master/API/description/api-description.pdf}.
\end{itemize}


\section{Input Documents}

The following documents have been the basis for the analysis, functional decomposition, and design of the openETCS OBU model:
\begin{itemize}
\item ERA Subset-026 \cite{subset-026}
\item ERA TSI CCS Documents
\item openETCS API documentation, available at \url{https://github.com/openETCS/modeling/blob/master/API/description/api-description.pdf}
\item openETCS requirements, i.e.~D2.1$\ldots$9, available at \url{https://github.com/openETCS/requirements/tree/master/Reference}
\todo[fancyline]{list has to be completed}
\end{itemize}


\section{Software and Tools used for Development}

The following software and tools have been used in the openETCS development process:
\todo[fancyline]{list and descriptions have to be checked for completeness}
\begin{description}
\item[SCADE System] Version 16.1b of SCADE System has been used for the the genereation of SysML models.
\item[SCADE Suite] Version 16.1b of SCADE Suite has been used for the functional modelling of the openETCS OBU components. Executable models are generated via the SCADE Suite code generator (KCG), which has been certified for CENELEC EN 50128 at SIL 3/4.
\item[SCADE Display] Version 16.1b of SCADE Display has been used for the development of the Driver Machine Interface (DMI).
\item[GitHub] The web based Git repository hosting service GitHub has been used for distributed revision control and source code respectively model management.
\end{description}


\section{General Remarks on the openETCS OBU Model}
The openETCS OBU model has been developed according the specification given in ERA Subset-026 \cite{subset-026}. The software release of the openETCS OBU documented and described in this document is publicly available at \url{https://github.com/openETCS/modeling/tree/master/model} and refers to the commit corresponding to hashtag 
\begin{quotation}
\texttt{1c06cc2d4a0d8f27569e065e2a9edf924b453ff1}
\end{quotation}
In particular, the root of the SCADE System SysML model is located at
\url{https://github.com/openETCS/modeling/tree/master/model/system}
and the root of the functional SCADE Suite model is located at
\url{https://github.com/openETCS/modeling/tree/master/model/Scade}.

Note that all components of the openETCS OBU have been developed from scratch, no existing components have been reused.