% This is an automatic generated Latex document based on the Iglos Glossary by Technische Universit�t Braunschweig. 
 %This Version has been released at the 11-Nov-2014 
 % new item : 
 % \newglossaryentry{?label?}{?key-val list?}
 %--------------------
 % usage in latex file
 %--------------------
%     Preamble
%--------------------
% \package[section, % add the glossary to the table of content
%            description,% acronyms have a user-supplied description,
 % style=superheaderborder, % table style
 % nonumberlist % no page number
 % ]{glossaries}
%\renewcommand*{\glossaryname}{List of Terms}
 %\makeglossaries
 %\loadglsentries{wp7_glossary}
 %--------------------
 %     reference
 %--------------------
 % \gls{label}
 %--------------------
 % compile command
 %--------------------
 % makeglossaries
 %--------------------);
 

 %Start of Glossary Terms
\newglossaryentry{application programming interface}{
 name={application programming interface},
 description={an abstraction that is defined by the description of an interface and the behaviour of the interface.}
 }
 
\newacronym{API} % label 
{API} %abbreviation 
 {application programming interface} %long form 
 
\newglossaryentry{balise group}{
 name={balise group},
 description={One or more balises which are treated as having the same reference location on the track.}
 }
 
\newacronym{BG} % label 
{BG} %abbreviation 
 {balise group} %long form 
 
\newglossaryentry{balise group message}{
 name={balise group message},
 description={}
 }
 
\newacronym{BGM} % label 
{BGM} %abbreviation 
 {balise group message} %long form 
 
\newglossaryentry{balise telegram}{
 name={balise telegram},
 description={A telegram contains one header and an identified and coherent set of packets. A message maybe comprised of one or several telegrams.}
 }
 
\newacronym{BT} % label 
{BT} %abbreviation 
 {balise telegram} %long form 
 
\newglossaryentry{Balise Transmission Module}{
 name={Balise Transmission Module},
 description={On board equipment for intermittent transmission between track and train. It shall be able to receive telegrams from a balise.}
 }
 
\newacronym{BTM} % label 
{BTM} %abbreviation 
 {Balise Transmission Module} %long form 
 
\newglossaryentry{Driver Machine Interface}{
 name={Driver Machine Interface},
 description={ERTMS train-borne device to enable communication between ETCS and/or GSM-R and the train driver.}
 }
 
\newacronym{DMI} % label 
{DMI} %abbreviation 
 {Driver Machine Interface} %long form 
 
\newglossaryentry{European Vital Computer}{
 name={European Vital Computer},
 description={Computer device for the onboard ETCS.}
 }
 
\newacronym{EVC} % label 
{EVC} %abbreviation 
 {European Vital Computer} %long form 
 
\newglossaryentry{EURORADIO}{
 name={EURORADIO},
 description={The functions required of a radio network coupled with the message protocols that provide an acceptably safe communications channel between track side and train borne equipment's}
 }
 
\newglossaryentry{Juridical Recording Unit}{
 name={Juridical Recording Unit},
 description={Device to record all actions and exchanges relating to the movement of trains sufficient for off line analysis of all events leading to an incident.}
 }
 
\newacronym{JRU} % label 
{JRU} %abbreviation 
 {Juridical Recording Unit} %long form 
 
\newglossaryentry{Last Relevant Balise Group}{
 name={Last Relevant Balise Group},
 description={It is the first balise group met and correctly read, when the linking information is not known by the train borne equipment. It is the last linked balise group found at the expected location and correctly read when the linking information is known by the train borne equipment. The LRBG is used as a common reference between the train borne and track side equipments in levels 2 and 3}
 }
 
\newacronym{LRBG} % label 
{LRBG} %abbreviation 
 {Last Relevant Balise Group} %long form 
 
\newglossaryentry{linking information}{
 name={linking information},
 description={Data defining the distance between groups of balises and the action to be taken if a balise group is not detected within given limits.}
 }
 
\newacronym{LI} % label 
{LI} %abbreviation 
 {linking information} %long form 
 
\newglossaryentry{location}{
 name={location},
 description={Location describes a position in terms of topological relations.}
 }
 
\newglossaryentry{Loop Transmission Module}{
 name={Loop Transmission Module},
 description={Train borne equipment that reads the track mounted loop data.}
 }
 
\newacronym{LTM} % label 
{LTM} %abbreviation 
 {Loop Transmission Module} %long form 
 
\newglossaryentry{odometry}{
 name={odometry},
 description={The process of measuring the train?s movement along the track. Used for speed measurement and distance measurement.}
 }
 
\newacronym{ODO} % label 
{ODO} %abbreviation 
 {odometry} %long form 
 
\newglossaryentry{on-board unit}{
 name={on-board unit},
 description={on-board equipment for ETCS and the ETCS-related GSM-R.}
 }
 
\newacronym{OBU} % label 
{OBU} %abbreviation 
 {on-board unit} %long form 
 
\newglossaryentry{orientation}{
 name={orientation},
 description={}
 }
 
\newglossaryentry{radio message}{
 name={radio message},
 description={The Radio Block Centre (RBC) sends electronic messages to, and receives electronic messages from, ETCS onboard equipment on trains within the area which the RBC is controlling. These messages are transmitted via GSM-R data radio}
 }
 
\newacronym{RM} % label 
{RM} %abbreviation 
 {radio message} %long form 
 
\newglossaryentry{service brake}{
 name={service brake},
 description={Train stopping, from a given speed, at such a deceleration that the passengers do not suffer discomfort or alarm or at an equivalent deceleration in the case of non-passenger trains.}
 }
 
\newacronym{SB} % label 
{SB} %abbreviation 
 {service brake} %long form 
 
\newglossaryentry{Specific Transmission Module}{
 name={Specific Transmission Module},
 description={The train borne equipment of the ERTMS / ETCS must be able to be interfaced with the train borne equipment of an existing train supervision system. The Specific Transmission Module shall perform a translation function between these systems and the ERTMS / ETCS.}
 }
 
\newacronym{STM} % label 
{STM} %abbreviation 
 {Specific Transmission Module} %long form 
 
\newglossaryentry{system requirement specification}{
 name={system requirement specification},
 description={Specification describing the technical properties of a piece of equipment based on a corresponding functional requirement specification.}
 }
 
\newacronym{SRS} % label 
{SRS} %abbreviation 
 {system requirement specification} %long form 
 
\newglossaryentry{Systems Modeling Language}{
 name={Systems Modeling Language},
 description={The Systems Modeling Language (SysML) is general purpose visual modeling language for systems engineering applications. SysML is defined as a dialect of the Unified Modeling Language (UML) standard, and supports the specification, analysis, design, verification and validation of a broad range of systems and systems-of-systems. These systems may include hardware, software, information, processes, personnel, and facilities.}
 }
 
\newacronym{SysML} % label 
{SysML} %abbreviation 
 {Systems Modeling Language} %long form 
 
\newglossaryentry{Train Interface Unit}{
 name={Train Interface Unit},
 description={The unit that provides the interface between the train borne equipment and the train. It is likely to be unique to a class of train.}
 }
 
\newacronym{TIU} % label 
{TIU} %abbreviation 
 {Train Interface Unit} %long form 
 
\newglossaryentry{train position}{
 name={train position},
 description={information related to the position of a train on the railway infrastructure.}
 }
 
\newacronym{TP} % label 
{TP} %abbreviation 
 {train position} %long form 
 

 %End of Glossary Terms
 %Glossary entries: 23 
 %Abbreviations entries: 20 
