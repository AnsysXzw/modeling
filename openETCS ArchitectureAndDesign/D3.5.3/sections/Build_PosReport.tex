%set the master document for easy compilation
%!TEX root = ../D3_5_3.tex

\paragraph{Component Requirements}

\begin{longtable}{p{.25\textwidth}p{.7\textwidth}}
\toprule
Component name			& Build\_PosReport \\
\midrule
Link to SCADE model		& {\footnotesize \url{https://github.com/openETCS/modeling/blob/master/model/Scade/System/ObuFunctions/ManageLocationRelatedInformation/TrainPosition/ProvidePositionReport/ProvidePositionReport_Pkg.xscade}} \\
\midrule
SCADE designer			& Christian Stahl, TWT \\
\midrule
Description				& This operator builds nine position report messages -- there can be up to nine errors, and for each error an individual report has to be sent.
The fold operator ensures that the first report is invalid if the first error is not present but there exists an error in the error field. In other words,
one valid report will be built. If the errorVector does not contain a single error, 
then at least one report needs to be built (if the operator is triggered). \\
\midrule
Input documents	& 
Subset-026, Chapter 3.6.5 \\
\midrule
Safety integrity level		& 4 \\
\midrule
Time constraints		& n/a
\\
\midrule
API requirements 		& n/a \\
\bottomrule
\end{longtable}


\paragraph{Interface}

For an overview of the interface of this internal component we refer to the SCADE model (cf.~link above) respectively the SCADE generated documentation.