%set the master document for easy compilation
%!TEX root = ../D3_5_2.tex

\paragraph{Component Requirements}

\begin{longtable}{p{.25\textwidth}p{.7\textwidth}}
\toprule
Component name			& SDM\_Commands \\
\midrule
Link to SCADE model		& {\footnotesize \url{???}} \\
\midrule
SCADE designer			& ??? \\
\midrule
Description				& This operator models the speed and distance monitoring commands. More precisely, it triggers the service or emergency brake and outputs the current supervision status of the OBU together with information on speeds and locations to the driver.

The OBU can be in any of three types of speed and distance monitoring modes: ceiling speed monitoring, release speed monitoring and target speed monitoring. We use a state machine to model the switching between the three modes: each state models a mode and a transition between to states is enabled if the condition two switch between the two corresponding modes is evaluated to true. In each mode, the OBU can be in up to five different supervision stati. The behavior of changing from one status to another is also modeled as a state machine. As a result, the model is a hierarchical state machine.\\
\midrule
Input documents	& 
Subset-026, Chapter 3.13.10: Speed and distance monitoring commands \\
\midrule
Safety integrity level		& 4 \\
\midrule
Time constraints		& [If applicable description of time constraints, otherwise n/a] \\
\midrule
API requirements 		& [If applicable description of API requirements, otherwise n/a] \\
\bottomrule
\end{longtable}


\paragraph{Interface}

For an overview of the interface of this internal component we refer to the SCADE model (c.f.~link above) respectively the SCADE generated documentation.