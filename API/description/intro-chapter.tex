\chapter{Introduction and context}

\section{Introduction}

This document describes the ``openETCS API''. The openETCS API is an
open, standardized, interface between a vendor specific platform and
the openETCS application software. More details on the software
architecture are given in chapter \ref{software-arch}.

\paragraph{Main objectives} The main objectives of this document are
to describe an API:
\begin{itemize}
\item Suitable for every partner of openETCS project;
\item Compatible with Vendor specific API through an ``Adaptation Layer'';
\item Making explicit all assumptions, including non-functional ones;
\item Language agnostic, allowing interfacing with software written
  in C, Ada or other programming languages;
\item Simple;
\item Fulfilling safety objectives;
\item Offering reasonable performance.
\end{itemize}


\section{References}

The requirements for this document are defined in \cite[§7.1]{D2.6-9} and
\cite{API-req-2} documents.

Source of information and constraints for this document are following
documents:
\begin{itemize}
\item Alstom API proposal defined in \cite{alstom-api},
  \cite{alstom-api-app-layer} and \cite{alstom-api-data-dict};
\item Comments on Alstom API proposal in \cite{alstom-api-comments};
\item The top level interfaces of \cite{sysml-model};
\item SCADE openETCS model requirements on run-time defined in §2.2
  and §3 of \cite{scade-modelling-guide};
\item ERSA Simulator API\cite{ERSA-api};
\item ETCS on-board interfaces as pointed out in Figure 1 of §2.5.3 in
  \cite{subset-026}.
\end{itemize}


\section{Identified issues}

This documents aims at reaching a consensus, satisfying relevant point
of view of all openETCS project's partners.

As of now, we have identified following issues on which no consensus
has been found (yet!):
\begin{enumerate}
\item Timing requirements are currently incompatible between SCADE
  model and Alstom's API definition
  \begin{enumerate}
  \item 	Sub-issue regarding cycle time definition
  \item 	Sub-issue regarding ordering of events
  \end{enumerate}
\item The data flow definition in Alstom's API proposal is sometimes
  too far from SRS
\item No agreement on the overall behaviour of Basic software and
  Application software: when data flows are exchanged, how they are
  stored, constraints on them, ...
  \begin{enumerate}
  \item No agreement on general software architecture
  \item What is in application Sw and generic software?
  \end{enumerate}
\item Define what is in or out of application (odometry, crc, ...)?
  Allow differentiation points between suppliers
\item No agreement on the way the various components are calling each
  other: order sequence, constraints, ...
\item How to manage physical units in unambiguous way?
\item The API description should distinguish abstract and concrete
  parts:
  \begin{enumerate}
  \item Abstract parts: Dataflows, Datatypes, Input/Output, ...
  \item Concrete parts: Given by value or reference, C or Ada
    language, where is made allocation and (possible) de-allocation,
    ...
  \end{enumerate}
\item For WP5 needs, it would be preferable to have two levels within
  the API:
  \begin{itemize}
  \item a low level where only ``conduits'' are opened between
    hardware architecture units. Opaque messages would be exchanged
    over such conduits;
  \item a higher level where the specific kind of information
    exchanged of the conduits is detailed (e.g. precise kind of
    DMI/EVC messages).
  \end{itemize}
  For the lower level, it seems interesting to define non-functional
  properties (latency bounds, message size bounds, ...).
\end{enumerate}

\section{Abbreviations}

Abbreviations not described in this section are described in
\cite{subset-023}.

\begin{description}
\item[\define{API}] Application Programming Interface. An interface
  between two parts of a system, describing aspects that the two parts
  shall be compliant with and keeping other points undetermined
\item [\define{JRU}] Juridical Recording Unit
\end{description}

% LocalWords:  Alstom API openETCS SysML DMI EVC
